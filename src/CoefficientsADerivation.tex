\documentclass[12pt,letterpaper,oneside,reqno]{amsart}
\usepackage{amsfonts}
\usepackage{amsmath}
\usepackage{amssymb}
\usepackage{amsthm}
\usepackage{float}
\usepackage{mathrsfs}
\usepackage{colonequals}
\usepackage[font=small,labelfont=bf]{caption}
\usepackage[left=1in,right=1in,bottom=1in,top=1in]{geometry}
\usepackage[pdfpagelabels,hyperindex,colorlinks=true,linkcolor=blue,urlcolor=magenta,citecolor=green]{hyperref}
\usepackage{graphicx}
\linespread{1.7}
\emergencystretch=1em

\newcommand \anglePower [2]{\langle #1 \rangle \sp{#2}}
\newcommand \bernoulli [2][B] {{#1}\sb{#2}}
\newcommand \curvePower [2]{\{#1\}\sp{#2}}
\newcommand \coeffA [3][A] {{\mathbf{#1}} \sb{#2,#3}}
\newcommand \polynomialP [4][P]{{\mathbf{#1}}\sp{#2} \sb{#3}(#4)}

% ordinary derivatives
\newcommand \derivative [2] {\frac{d}{d #2} #1}                              % 1 - function; 2 - variable;
\newcommand \pderivative [2] {\frac{\partial #1}{\partial #2}}               % 1 - function; 2 - variable;
\newcommand \qderivative [1] {D_{q} #1}                                      % 1 - function
\newcommand \nqderivative [1] {D_{n,q} #1}                                   % 1 - function
\newcommand \qpowerDerivative [1] {\mathcal{D}_q #1}                         % 1 - function;
\newcommand \finiteDifference [1] {\Delta #1}                                % 1 - function;
\newcommand \pTsDerivative [2] {\frac{\partial #1}{\Delta #2}}               % 1 - function; 2 - variable;

% high order derivatives
\newcommand \derivativeHO [3] {\frac{d^{#3}}{d {#2}^{#3}} #1}                % 1 - function; 2 - variable; 3 - order
\newcommand \pderivativeHO [3]{\frac{\partial^{#3}}{\partial {#2}^{#3}} #1}
\newcommand \qderivativeHO [2] {D_{q}^{#2} #1}                               % 1 - function; 2 - order
\newcommand \qpowerDerivativeHO [2] {\mathcal{D}_{q}^{#2} #1}                % 1 - function; 2 - order
\newcommand \finiteDifferenceHO [2] {\Delta^{#2} #1}                         % 1 - function; 2 - order
\newcommand \pTsDerivativeHO [3] {\frac{\partial^{#3}}{\Delta {#2}^{#3}} #1} % 1 - function; 2 - variable;

\newtheorem{thm}{Theorem}[section]
\newtheorem{cor}[thm]{Corollary}
\newtheorem{lem}[thm]{Lemma}
\newtheorem{examp}[thm]{Example}

\numberwithin{equation}{section}

\title[Derivation of A Coefficients]
{Derivation of $\coeffA{m}{r}$ Coefficients}
\author[Petro Kolosov]{Petro Kolosov}
\email{kolosovp94@gmail.com}
\keywords{
    Binomial theorem, Polynomials, Faulhaber's formula
}
\urladdr{https://kolosovpetro.github.io}
\subjclass[2010]{26E70, 05A30}
\date{\today}
\hypersetup{
    pdftitle={Derivation of A Coefficients},
    pdfsubject={Discrete Mathematics, Number Theory, Combinatorics},
    pdfauthor={Petro Kolosov},
    pdfkeywords={Binomial theorem, Discrete convolution, Power function, Polynomials, Convolution,
    Multinomial theorem, Binomial coefficient, Bernoulli number, Pascal's triangle, Faulhaber's formula,
    Power sum, Worpitzky identity, Binomial expansion}
}
\begin{document}
    \begin{abstract}
        Derivation of $\coeffA{m}{r}$ in a simple and explicit manner.
    \end{abstract}

    \maketitle

    \tableofcontents


    \section{Introduction and Main Results} \label{sec:introduction}
    Considering the table of forward finite differences of the polynomial $n^3$
\begin{table}[H]
    \begin{center}
        \setlength\extrarowheight{-6pt}
        \begin{tabular}{c|cccc}
            $n$ & $n^3$ & $\Delta(n^3)$ & $\Delta^2(n^3)$ & $\Delta^3(n^3)$ \\
            \hline
            0   & 0     & 1             & 6               & 6               \\
            1   & 1     & 7             & 12              & 6               \\
            2   & 8     & 19            & 18              & 6               \\
            3   & 27    & 37            & 24              & 6               \\
            4   & 64    & 61            & 30              & 6               \\
            5   & 125   & 91            & 36              &                 \\
            6   & 216   & 127           &                 &                 \\
            7   & 343   &               &                 &
        \end{tabular}
    \end{center}
    \caption{Table of finite differences of the polynomial $n^3$.} \label{tab:table}
\end{table}
We can easily observe that finite differences
\footnote{One may assume that it is possible to reach the form $n^{2m+1} = \sum_{k=1}^{n} \mathbf{A}_{m,0} k^0 (n-k)^0 + \mathbf{A}_{m,1}(n-k)^1
+ \cdots + \mathbf{A}_{m,m} k^m (n-k)^m$ simply taking finite differences of the polynomial $n^{2m+1}$ up to order of $2m+1$ 
and interpolating it backwards similarly as shown in~\eqref{eq:cubes_interpolation}.
However, my observations do not provide any evidence of such assumption.
Interestingly enough is that we could have been arrived to the pure differential approach of the relation~\eqref{eq:odd_power_conjecture} then.}
of the polynomial $n^3$ may be expressed according
to the following relation, via rearrangement of the terms
\begin{align}
    \label{eq:cubes_interpolation}
    \begin{split}
        \Delta(0^3) &= 1+6 \cdot 0 \\
        \Delta(1^3) &= 1+6\cdot0+6\cdot1 \\
        \Delta(2^3) &= 1+6\cdot0+6\cdot1+6\cdot2 \\
        \Delta(3^3) &= 1+6\cdot0+6\cdot1+6\cdot2+6\cdot3 \\
        &\; \; \vdots \\
        \Delta(n^3) &= 1+6\cdot0+6\cdot1+6\cdot2+6\cdot3+\cdots+6\cdot n
    \end{split}
\end{align}
Furthermore, the polynomial $n^3$ is equivalent to
\begin{align*}
    n^3 &= [1+6\cdot0]+[1+6\cdot0+6\cdot1]+[1+6\cdot0+6\cdot1+6\cdot2]+\cdots \\
    &+[1+6\cdot0+6\cdot1+6\cdot2+\cdots+6\cdot(n-1)]
\end{align*}
Rearranging the above equation, we get
\[
    n^3 = n +(n-0) \cdot6 \cdot0 + (n-1)\cdot6\cdot1 + (n-2)\cdot6\cdot2 + \cdots+1\cdot6\cdot(n-1)
\]
Therefore, we can consider the polynomial $n^3$ as
\begin{equation}
    \label{eq:cube_identity}
    n^3 = \sum_{k=1}^{n} 6k(n-k) + 1
\end{equation}
Assume that equation~\eqref{eq:cube_identity} has the following implicit form
\begin{equation}
    \label{eq:pattern}
    n^3 = \sum_{k=1}^{n} \coeffA{1}{1} k^1(n-k)^1 + \coeffA{1}{0} k^0(n-k)^0,
\end{equation}
where $\coeffA{1}{1} = 6$ and $\coeffA{1}{0} = 1$, respectively.
Note that here the power of $3$ is actually defined by $2m+1$ where $m=1$.
So, is there a generalization of the relation~\eqref{eq:pattern} for all positive odd powers $2m+1, \; m=0,1,2,\dots$?
Therefore, let us propose a conjecture
\begin{conj}
    For every $n\geq 1, \; n,m\in\mathbb{N}$ there are coefficients $\coeffA{m}{0}, \coeffA{m}{1}, \ldots, \coeffA{m}{m}$ such that
    \begin{equation}
        \label{eq:odd_power_conjecture}
        n^{2m+1} = \sum_{k=1}^{n} \coeffA{m}{0} k^0 (n-k)^0 + \coeffA{m}{1} (n-k)^1
        + \cdots + \coeffA{m}{m} k^m (n-k)^m
    \end{equation}
\end{conj}



    \section{Examples}\label{sec:examples}
    Let be an example for $m=2$ of
\begin{equation*}
    \sum_{r=0}^{m} \coeffA{m}{r} \left[ \frac{1}{(2r+1) \binom{2r}r} n^{2r+1} \right]
    + 2 \sum_{r=0}^{m} \coeffA{m}{r} \left[ \sum_{d=0}^{(r-1)/2} \frac{(-1)^r}{2r-2d} \binom{r}{2d+1} \bernoulli{2r-2d} n^{2d+1} \right]
    - n^{2m+1} = 0
\end{equation*}
So that
\begin{equation*}
    \sum_{r=0}^{2} \coeffA{2}{r} \left[ \frac{1}{(2r+1) \binom{2r}r} n^{2r+1} \right]
    + 2 \sum_{r=0}^{2} \coeffA{2}{r} \left[ \sum_{d=0}^{(r-1)/2} \frac{(-1)^r}{2r-2d} \binom{r}{2d+1} \bernoulli{2r-2d} n^{2d+1} \right]
    - n^{5} = 0
\end{equation*}
The sum $\sum_{r=0}^{2} \coeffA{2}{r} \left[ \frac{1}{(2r+1) \binom{2r}r} n^{2r+1} \right]$ in explicit form is
\begin{equation*}
    \sum_{r=0}^{2} \coeffA{2}{r} \left[ \frac{1}{(2r+1) \binom{2r}r} n^{2r+1} \right] =
    \coeffA{2}{0} n
    + \coeffA{2}{1} \left[ \frac{1}{3 \binom{2}{1}} n^{3} \right]
    + \coeffA{2}{2} \left[ \frac{1}{5 \binom{4}{2}} n^{5} \right]
\end{equation*}
Also the sum
$\sum_{r=0}^{2} \coeffA{2}{r} \left[ \sum_{d=0}^{(r-1)/2} \frac{(-1)^r}{2r-2d} \binom{r}{2d+1} \bernoulli{2r-2d} n^{2d+1} \right]$
is
\begin{equation*}
    \begin{split}
        &\sum_{r=0}^{2} \coeffA{2}{r} \left[ \sum_{d=0}^{(r-1)/2} \frac{(-1)^r}{2r-2d} \binom{r}{2d+1} \bernoulli{2r-2d} n^{2d+1} \right] = \\
        & + \coeffA{2}{0} \left[ \sum_{d=0}^{-1} \frac{1}{-2d} \binom{0}{2d+1} \bernoulli{-2d} n^{2d+1} \right] \\
        & - \coeffA{2}{1} \left[ \sum_{d=0}^{0} \frac{1}{2-2d} \binom{1}{2d+1} \bernoulli{2-2d} n^{2d+1} \right] \\
        & + \coeffA{2}{2} \left[ \sum_{d=0}^{0} \frac{1}{4-2d} \binom{2}{2d+1} \bernoulli{4-2d} n^{2d+1} \right] \\
        & = \coeffA{2}{0} \cdot 0
        - \coeffA{2}{1} \left[ \frac{1}{2} \binom{1}{1} \bernoulli{2} n^{1} \right]
        + \coeffA{2}{2} \left[ \frac{1}{4} \binom{2}{1} \bernoulli{4} n^{1} \right] \\
        & = \coeffA{2}{0} \cdot 0
        - \coeffA{2}{1} \left[ \frac{1}{12} n \right]
        - \coeffA{2}{2} \left[ \frac{1}{60} n \right] \\
        & = - \coeffA{2}{1} \frac{1}{12} n
        - \coeffA{2}{2}  \frac{1}{60} n
    \end{split}
\end{equation*}



    \section{Questions}\label{sec:questions}
    \begin{enumerate}
    \item Any proof or reference to the relation~\eqref{eq:combinatorial-identity}?
    \item What is the motivation to use~\eqref{eq:faulhaber-formula} version of Faulhaber's formula?
    \item Why is there twice odd $\ell$ in~\eqref{eq:polynomial-sum-1}?
\end{enumerate}


%    \section{Conclusions}\label{sec:conclusions}
%    In this manuscript, we have shown that for every $n\geq 1, \; n,m\in\mathbb{N}$
there are coefficients $\mathbf{A}_{m,0}, \mathbf{A}_{m,1}, \ldots, \mathbf{A}_{m,m}$ such that
the polynomial identity holds
\[
    n^{2m+1} = \sum_{k=1}^{n} \mathbf{A}_{m,0} k^0 (n-k)^0 + \mathbf{A}_{m,1}(n-k)^1
    + \cdots + \mathbf{A}_{m,m} k^m (n-k)^m
\]
In particular, the coefficients $\coeffA{m}{r}$ can be evaluated in both ways,
by constructing and solving a certain system of linear equations or by deriving a recurrence relation;
all these approaches are examined providing examples
in the sections~\eqref{sec:approach-via-system-of-linear-equations} and~\eqref{sec:finding-a-recurrence-relation}.
Moreover, to validate the results, supplementary Mathematica programs are available at~\cite{kolosov2023github}.

%
%    \bibliographystyle{unsrt}
%    \bibliography{CoefficientsADerivationReferences}

\end{document}
