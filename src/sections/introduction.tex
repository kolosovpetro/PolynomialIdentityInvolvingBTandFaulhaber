In this section the MathOverflow answer \href{https://mathoverflow.net/a/297916}{\texttt{mathoverflow.net/a/297916}}
is considered and analyzed precisely.
This section is copy-paste of original MO answer with in-place clarifications of some parts where formulae derivation is
not so obvious.
Few part are not clear for me at all, so that I kept original text there.
In general, this section is motivated to provide original answer and only after that there will be questions provided in
ongoing sections.
To clarify and simplify navigation over the document please find table of contents.
All equations in this document are numbered (regardless if they are referenced or not), so that if you have some comment
it is simpler to reference particular formula.
So, let's begin.
Consider the polynomial relation
\begin{equation}
    n^{2m+1} = \sum_{r=0}^{m} \coeffA{m}{r} \sum_{k=1}^{n} k^{r} (n-k)^{r}\label{eq:odd-power-identity}
\end{equation}
Expanding the $(n-k)^r$ part via Binomial theorem we get
\begin{equation}
    \begin{split}
        n^{2m+1} &= \sum_{r=0}^{m} \coeffA{m}{r} \sum_{k=1}^{n} k^{r} (n-k)^{r} \\
        &= \sum_{r=0}^{m} \coeffA{m}{r} \sum_{k=1}^{n} k^{r} \left[ \sum_{t=0}^{r} (-1)^t \binom{r}{t} n^{r-t} k^{t} \right] \\
        &= \sum_{r=0}^{m} \coeffA{m}{r} \left[ \sum_{t=0}^{r} (-1)^t \binom{r}{t} n^{r-t} \sum_{k=1}^{n} k^{t+r} \right] \\
    \end{split}\label{eq:equation2}
\end{equation}
Consider the Faulhaber's formula
\begin{equation}
    \sum_{k=1}^{n} k^{p} = \frac{1}{p+1}\sum_{j=0}^{p} \binom{p+1}{j} \bernoulli{j} n^{p+1-j}\label{eq:equation3}
\end{equation}
it is very important to note that summation bound is $p$ while binomial coefficient upper bound is $p+1$.
It means that we cannot skip summation bounds unless we do some trick as
\begin{equation}
    \begin{split}
        \sum_{k=1}^{n} k^{p} &= \frac{1}{p+1}\sum_{j=0}^{p} \binom{p+1}{j} \bernoulli{j} n^{p+1-j} \\
        &= \left[ \frac{1}{p+1}\sum_{j=0}^{p+1} \binom{p+1}{j} \bernoulli{j} n^{p+1-j} \right] - \bernoulli{p+1} \\
        &= \left[ \frac{1}{p+1}\sum_{j} \binom{p+1}{j} \bernoulli{j} n^{p+1-j} \right] - \bernoulli{p+1} \\
    \end{split}\label{eq:faulhaber-formula}
\end{equation}
Using Faulhaber's formula
$\sum_{k=1}^{n} k^{p} = \left[ \frac{1}{p+1}\sum_{j} \binom{p+1}{j} \bernoulli{j} n^{p+1-j} \right] - \bernoulli{p+1}$
we get
\begin{equation}
    \begin{split}
        \sum_{k=1}^{n} k^{r} (n-k)^{r}
        &= \sum_{t=0}^{r} (-1)^t \binom{r}{t} n^{r-t} \sum_{k=1}^{n} k^{t+r} \\
        &= \sum_{t=0}^{r} (-1)^t \binom{r}{t} n^{r-t} \left[ \frac{1}{t+r+1} \sum_{j} \binom{t+r+1}{j} \bernoulli{j} n^{t+r+1-j} - \bernoulli{t+r+1} \right] \\
        &= \sum_{t=0}^{r} \binom{r}{t} \left[ \frac{(-1)^t}{t+r+1} \sum_{j} \binom{t+r+1}{j} \bernoulli{j} n^{2r+1-j} - \bernoulli{t+r+1} n^{r-t} \right] \\
        &= \sum_{t=0}^{r} \binom{r}{t} \frac{(-1)^t}{t+r+1} \sum_{j} \binom{t+r+1}{j} \bernoulli{j} n^{2r+1-j} - \sum_{t=0}^{r} \binom{r}{t} \frac{(-1)^t}{t+r+1} \bernoulli{t+r+1} n^{r-t} \\
        &= \sum_{j} \sum_{t} \binom{r}{t} \frac{(-1)^t}{t+r+1} \binom{t+r+1}{j} \bernoulli{j} n^{2r+1-j} - \sum_{t=0}^{r} \binom{r}{t} \frac{(-1)^t}{t+r+1} \bernoulli{t+r+1} n^{r-t} \\
        &= \sum_{j} \bernoulli{j} n^{2r+1-j} \sum_{t} \binom{r}{t} \frac{(-1)^t}{t+r+1} \binom{t+r+1}{j} - \sum_{t=0}^{r} \binom{r}{t} \frac{(-1)^t}{t+r+1} \bernoulli{t+r+1} n^{r-t}
    \end{split}\label{eq:equation4}
\end{equation}
Now, we notice that
\begin{equation}
    \sum_{t} \binom{r}{t} \frac{(-1)^t}{r+t+1} \binom{r+t+1}{j}
    =\begin{cases}
         \frac{1}{(2r+1) \binom{2r}r}, & \text{if } j=0;\\
         \frac{(-1)^r}{j} \binom{r}{2r-j+1}, & \text{if } j>0.
    \end{cases}\label{eq:combinatorial-identity}
\end{equation}
In particular, the last sum is zero for $0< t \leq j$.
So taking $j=0$ we have
\begin{equation}
    \begin{split}
        \sum_{k=1}^{n} k^{r} (n-k)^{r}
        &= \frac{1}{(2r+1) \binom{2r}r} n^{2r+1} + \left[ \sum_{j \geq 1} \bernoulli{j} n^{2r+1-j} \sum_{t} \binom{r}{t} \frac{(-1)^t}{t+r+1} \binom{t+r+1}{j} \right] \\
        &- \left[ \sum_{t=0}^{r} \binom{r}{t} \frac{(-1)^t}{t+r+1} \bernoulli{t+r+1} n^{r-t} \right]
    \end{split}\label{eq:equation5}
\end{equation}
Now let's simplify the double summation
\begin{equation}
    \begin{split}
        \sum_{k=1}^{n} k^{r} (n-k)^{r}
        &= \frac{1}{(2r+1) \binom{2r}r} n^{2r+1}
        + \underbrace{\left[ \sum_{j \geq 1} \frac{(-1)^r}{j} \binom{r}{2r-j+1} \bernoulli{j} n^{2r+1-j} \right]}_{(\star)} \\
        &- \underbrace{\left[ \sum_{t=0}^{r} \binom{r}{t} \frac{(-1)^t}{t+r+1} \bernoulli{t+r+1} n^{r-t} \right]}_{(\diamond)}
    \end{split}\label{eq:equation6}
\end{equation}
Hence, introducing $\ell=2r-j+1$ to $(\star)$ and $\ell=r-t$ to $(\diamond)$, we get
\begin{equation}
    \begin{split}
        \sum_{k=1}^{n} k^{r} (n-k)^{r}
        &= \frac{1}{(2r+1) \binom{2r}r} n^{2r+1}
        + \left[ \sum_{\ell} \frac{(-1)^r}{2r+1-\ell} \binom{r}{\ell} \bernoulli{2r+1-\ell} n^{\ell} \right] \\
        &- \left[ \sum_{\ell} \binom{r}{\ell} \frac{(-1)^{r-\ell}}{2r+1-\ell} \bernoulli{2r+1-\ell} n^{\ell} \right]\\
        &= \frac{1}{(2r+1) \binom{2r}r} n^{2r+1} + 2 \sum_{\mathrm{odd \; \ell}} \frac{(-1)^r}{2r+1-\ell} \binom{r}{\ell} \bernoulli{2r+1-\ell} n^{\ell}
    \end{split}\label{eq:polynomial-sum-1}
\end{equation}
Using the definition of $\coeffA{m}{r}$, we obtain the following identity for polynomials in $n$
\begin{equation}
    \label{eq:equation}
    \sum_{r} \coeffA{m}{r} \frac{1}{(2r+1) \binom{2r}r} n^{2r+1}
    + 2 \sum_{r} \coeffA{m}{r} \sum_{\mathrm{odd \; \ell}} \frac{(-1)^r}{2r+1-\ell} \binom{r}{\ell} \bernoulli{2r+1-\ell} n^{\ell}
    \equiv n^{2m+1}
\end{equation}
Replacing odd $\ell$ by $d$ we get
\begin{equation}
    \begin{split}
        &\sum_{r} \coeffA{m}{r} \frac{1}{(2r+1) \binom{2r}r} n^{2r+1}
        + 2 \sum_{r} \coeffA{m}{r} \sum_{d} \frac{(-1)^r}{2r-2d} \binom{r}{2d+1} \bernoulli{2r-2d} n^{2d+1}
        \equiv n^{2m+1} \\
        &\sum_{r} \coeffA{m}{r} \left[ \frac{1}{(2r+1) \binom{2r}r} n^{2r+1} \right]
        + 2 \sum_{r} \coeffA{m}{r} \left[ \sum_{d} \frac{(-1)^r}{2r-2d} \binom{r}{2d+1} \bernoulli{2r-2d} n^{2d+1} \right]
        - n^{2m+1} = 0.
    \end{split}\label{eq:equation7}
\end{equation}
Taking the coefficient of $n^{2m+1}$ in~\eqref{eq:equation7}, we get
\begin{equation}
    \coeffA{m}{m} = (2m+1)\binom{2m}{m}\label{eq:equation8}
\end{equation}
and taking the coefficient of $x^{2d+1}$ for an integer $d$ in the range $m/2 \leq d < m$, we get
\begin{equation}
    \coeffA{m}{d} = 0\label{eq:equation9}
\end{equation}
Taking the coefficient of $n^{2d+1}$ for $d$ in the range $m/4 \leq d < m/2$ we get
\begin{equation}
    \coeffA{m}{d} \frac{1}{(2d+1) \binom{2d}{d}}
    +2 \underbrace{(2m+1) \binom{2m}{m}}_{\coeffA{m}{m}} \binom{m}{2d+1} \frac{(-1)^m}{2m-2d} \bernoulli{2m-2d} = 0,\label{eq:equation10}
\end{equation}
i.e
\begin{equation}
    \coeffA{m}{d} = (-1)^{m-1} \frac{(2m+1)!}{d!d!m!(m-2d-1)!} \frac{1}{m-d} \bernoulli{2m-2d}\label{eq:equation11}
\end{equation}
Continue similarly we can express $\coeffA{m}{r}$ for each integer $r$ in range $m/2^{s+1}\leq r < m/2^s$
(iterating consecutively $s=1,2,\ldots$) via previously determined values of $\coeffA{m}{d}$ as follows
\begin{equation}
    \coeffA{m}{r} =
    (2r+1) \binom{2r}{r} \sum_{d=2r+1}^{m} \coeffA{m}{d} \binom{d}{2r+1} \frac{(-1)^{d-1}}{d-r}
    \bernoulli{2d-2r}\label{eq:equation12}
\end{equation}
Finally, the coefficient $\coeffA{m}{r}$ is defined recursively as
\begin{equation}
    \label{eq:def_coeff_a}
    \coeffA{m}{r} \colonequals
    \begin{cases}
    (2r+1)
        \binom{2r}{r}, & \text{if } r=m; \\
        (2r+1) \binom{2r}{r} \sum_{d=2r+1}^{m} \coeffA{m}{d} \binom{d}{2r+1} \frac{(-1)^{d-1}}{d-r}
        \bernoulli{2d-2r}, & \text{if } 0 \leq r<m; \\
        0, & \text{if } r<0 \text{ or } r>m,
    \end{cases}
\end{equation}
where $\bernoulli{t}$ are Bernoulli numbers.
It is assumed that $\bernoulli{1}=\frac{1}{2}$.
%Consider the case $r=m$
%\begin{equation*}
%    \coeffA{m}{m} \left[ \frac{1}{(2m+1) \binom{2m}{m}} n^{2r+1} \right]
%    + 2 \coeffA{m}{m} \left[ \sum_{d} \frac{(-1)^m}{2m-2d} \binom{m}{2d+1} \bernoulli{2m-2d} n^{2d+1} \right]
%    - n^{2m+1} = 0.
%\end{equation*}
%\begin{equation}
%    \label{eq:proof1}
%    \begin{split}
%        \sum_{k=1}^{n} k^r (n-k)^r
%        &=\sum_{j=0}^{r} \binom{r}{j} n^{r-j} \frac{(-1)^j}{r+j+1}
%        \left[\sum_{s} \binom{r+j+1}{s} \bernoulli{s} n^{r+j+1-s} - \bernoulli{r+j+1} \right] \\
%        &=\sum_{j,s} \binom{r}{j} \frac{(-1)^j}{r+j+1} \binom{r+j+1}{s} \bernoulli{s} n^{2r+1-s}
%        -\sum_{j} \binom{r}{j} \frac{(-1)^j}{r+j+1} \bernoulli{r+j+1} n^{r-j} \\
%        &=\sum_{s} \underbrace{\sum_{j} \binom{r}{j} \frac{(-1)^j}{r+j+1} \binom{r+j+1}{s}}_{S(r)}
%        \bernoulli{s} n^{2r+1-s} \\
%        &-\sum_{j} \binom{r}{j} \frac{(-1)^j}{r+j+1} \bernoulli{r+j+1} n^{r-j}
%    \end{split}
%\end{equation}