Another approach to determine the coefficients $\coeffA{m}{r}$ was provided by Dr. Max Alekseyev
in MathOverflow discussion~\cite{alekseyev2018mathoverflow}.
Generally, the idea was to determine the coefficients $\coeffA{m}{r}$ recursively starting from the base case
$\coeffA{m}{m}$ up to $\coeffA{m}{r-1}, \ldots, \coeffA{m}{0}$ via previously determined values.
Consider the Faulhaber's formula
\begin{equation*}
    \sum_{k=1}^{n} k^{p} = \frac{1}{p+1}\sum_{j=0}^{p} \binom{p+1}{j} \bernoulli{j} n^{p+1-j}
\end{equation*}
it is very important to note that summation bound is $p$ while binomial coefficient upper bound is $p+1$.
It means that we cannot skip summation bounds unless we do some trick as
\begin{equation*}
    \begin{split}
        \sum_{k=1}^{n} k^{p}
        = \frac{1}{p+1}\sum_{j=0}^{p} \binom{p+1}{j} \bernoulli{j} n^{p+1-j}
        &= \left[ \frac{1}{p+1}\sum_{j=0}^{p+1} \binom{p+1}{j} \bernoulli{j} n^{p+1-j} \right] - \bernoulli{p+1} \\
        &= \left[ \frac{1}{p+1}\sum_{j} \binom{p+1}{j} \bernoulli{j} n^{p+1-j} \right] - \bernoulli{p+1}
    \end{split}
\end{equation*}
Using the Faulhaber's formula
$\sum_{k=1}^{n} k^{p} = \left[ \frac{1}{p+1}\sum_{j} \binom{p+1}{j} \bernoulli{j} n^{p+1-j} \right] - \bernoulli{p+1}$
we get
\begin{equation*}
    \begin{split}
        \sum_{k=1}^{n} k^{r} (n-k)^{r}
        &= \sum_{t=0}^{r} (-1)^t \binom{r}{t} n^{r-t} \sum_{k=1}^{n} k^{t+r} \\
        &= \sum_{t=0}^{r} (-1)^t \binom{r}{t} n^{r-t} \left[ \frac{1}{t+r+1} \sum_{j} \binom{t+r+1}{j} \bernoulli{j} n^{t+r+1-j} - \bernoulli{t+r+1} \right] \\
        &= \sum_{t=0}^{r} \binom{r}{t} \left[ \frac{(-1)^t}{t+r+1} \sum_{j} \binom{t+r+1}{j} \bernoulli{j} n^{2r+1-j} - \bernoulli{t+r+1} n^{r-t} \right] \\
        &= \sum_{t=0}^{r} \binom{r}{t} \frac{(-1)^t}{t+r+1} \sum_{j} \binom{t+r+1}{j} \bernoulli{j} n^{2r+1-j} - \sum_{t=0}^{r} \binom{r}{t} \frac{(-1)^t}{t+r+1} \bernoulli{t+r+1} n^{r-t} \\
        &= \sum_{j} \sum_{t} \binom{r}{t} \frac{(-1)^t}{t+r+1} \binom{t+r+1}{j} \bernoulli{j} n^{2r+1-j} - \sum_{t=0}^{r} \binom{r}{t} \frac{(-1)^t}{t+r+1} \bernoulli{t+r+1} n^{r-t} \\
        &= \sum_{j} \bernoulli{j} n^{2r+1-j} \sum_{t} \binom{r}{t} \frac{(-1)^t}{t+r+1} \binom{t+r+1}{j} - \sum_{t=0}^{r} \binom{r}{t} \frac{(-1)^t}{t+r+1} \bernoulli{t+r+1} n^{r-t}
    \end{split}
\end{equation*}
Now, we notice that
\begin{equation}
    \sum_{t} \binom{r}{t} \frac{(-1)^t}{r+t+1} \binom{r+t+1}{j}
    =\begin{cases}
         \frac{1}{(2r+1) \binom{2r}r}, & \text{if } j=0;\\
         \frac{(-1)^r}{j} \binom{r}{2r-j+1}, & \text{if } j>0.
    \end{cases}\label{eq:combinatorial-identity}
\end{equation}
An elegant proof of the above binomial identity is provided at~\cite{scheuer2023mathstackexchange}.
In particular, the equation~\eqref{eq:combinatorial-identity} is zero for $0< t \leq j$.
So that taking $j=0$ we have
\begin{equation*}
    \begin{split}
        \sum_{k=1}^{n} k^{r} (n-k)^{r}
        &= \frac{1}{(2r+1) \binom{2r}r} n^{2r+1} + \left[ \sum_{j \geq 1} \bernoulli{j} n^{2r+1-j} \sum_{t} \binom{r}{t} \frac{(-1)^t}{t+r+1} \binom{t+r+1}{j} \right] \\
        &- \left[ \sum_{t=0}^{r} \binom{r}{t} \frac{(-1)^t}{t+r+1} \bernoulli{t+r+1} n^{r-t} \right]
    \end{split}
\end{equation*}
Now let's simplify the double summation applying the identity~\eqref{eq:combinatorial-identity}
\begin{equation*}
    \begin{split}
        \sum_{k=1}^{n} k^{r} (n-k)^{r}
        &= \frac{1}{(2r+1) \binom{2r}r} n^{2r+1}
        + \underbrace{\left[ \sum_{j \geq 1} \frac{(-1)^r}{j} \binom{r}{2r-j+1} \bernoulli{j} n^{2r+1-j} \right]}_{(\star)} \\
        &- \underbrace{\left[ \sum_{t=0}^{r} \binom{r}{t} \frac{(-1)^t}{t+r+1} \bernoulli{t+r+1} n^{r-t} \right]}_{(\diamond)}
    \end{split}
\end{equation*}
Hence, introducing $\ell=2r-j+1$ to $(\star)$ and $\ell=r-t$ to $(\diamond)$ we collapse the common terms of the above
equation so that we get
\begin{equation*}
    \begin{split}
        \sum_{k=1}^{n} k^{r} (n-k)^{r}
        &= \frac{1}{(2r+1) \binom{2r}r} n^{2r+1}
        + \left[ \sum_{\ell} \frac{(-1)^r}{2r+1-\ell} \binom{r}{\ell} \bernoulli{2r+1-\ell} n^{\ell} \right] \\
        &- \left[ \sum_{\ell} \binom{r}{\ell} \frac{(-1)^{r-\ell}}{2r+1-\ell} \bernoulli{2r+1-\ell} n^{\ell} \right]\\
        &= \frac{1}{(2r+1) \binom{2r}r} n^{2r+1} + 2 \sum_{\mathrm{odd \; \ell}} \frac{(-1)^r}{2r+1-\ell} \binom{r}{\ell} \bernoulli{2r+1-\ell} n^{\ell}
    \end{split}
\end{equation*}
Using the definition of $\coeffA{m}{r}$, we obtain the following identity for polynomials in $n$
\begin{equation*}
    \sum_{r} \coeffA{m}{r} \frac{1}{(2r+1) \binom{2r}r} n^{2r+1}
    + 2 \sum_{r} \coeffA{m}{r} \sum_{\mathrm{odd \; \ell}} \frac{(-1)^r}{2r+1-\ell} \binom{r}{\ell} \bernoulli{2r+1-\ell} n^{\ell}
    \equiv n^{2m+1}
\end{equation*}
Replacing odd $\ell$ by $d$ we get
\begin{equation}
    \begin{split}
        &\sum_{r} \coeffA{m}{r} \frac{1}{(2r+1) \binom{2r}r} n^{2r+1}
        + 2 \sum_{r} \coeffA{m}{r} \sum_{d} \frac{(-1)^r}{2r-2d} \binom{r}{2d+1} \bernoulli{2r-2d} n^{2d+1}
        \equiv n^{2m+1} \\
        &\sum_{r} \coeffA{m}{r} \left[ \frac{1}{(2r+1) \binom{2r}r} n^{2r+1} \right]
        + 2 \sum_{r} \coeffA{m}{r} \left[ \sum_{d} \frac{(-1)^r}{2r-2d} \binom{r}{2d+1} \bernoulli{2r-2d} n^{2d+1} \right]
        - n^{2m+1} = 0 \\
        &\sum_{r=0}^{m} \coeffA{m}{r} \left[ \frac{1}{(2r+1) \binom{2r}r} n^{2r+1} \right]
        + 2 \sum_{r=0}^{m} \coeffA{m}{r} \left[ \sum_{d=0}^{(r-1)/2} \frac{(-1)^r}{2r-2d} \binom{r}{2d+1} \bernoulli{2r-2d} n^{2d+1} \right]
        - n^{2m+1} = 0 \\
    \end{split}\label{eq:equation7}
\end{equation}
Taking the coefficient of $n^{2m+1}$ in~\eqref{eq:equation7}, we get
\begin{equation*}
    \coeffA{m}{m} = (2m+1)\binom{2m}{m}
\end{equation*}
and taking the coefficient of $n^{2d+1}$ for an integer $d$ in the range $m/2 \leq d < m$, we get
\begin{equation*}
    \coeffA{m}{d} = 0
\end{equation*}
Taking the coefficient of $n^{2d+1}$ for $d$ in the range $m/4 \leq d < m/2$ we get
\begin{equation*}
    \coeffA{m}{d} \frac{1}{(2d+1) \binom{2d}{d}}
    +2 (2m+1) \binom{2m}{m}\binom{m}{2d+1} \frac{(-1)^m}{2m-2d} \bernoulli{2m-2d} = 0
\end{equation*}
i.e
\begin{equation*}
    \coeffA{m}{d} = (-1)^{m-1} \frac{(2m+1)!}{d!d!m!(m-2d-1)!} \frac{1}{m-d} \bernoulli{2m-2d}
\end{equation*}
Continue similarly we can express $\coeffA{m}{r}$ for each integer $r$ in range $m/2^{s+1}\leq r < m/2^s$
(iterating consecutively $s=1,2,\ldots$) via previously determined values of $\coeffA{m}{d}$ as follows
\begin{equation*}
    \coeffA{m}{r} =
    (2r+1) \binom{2r}{r} \sum_{d \geq 2r+1}^{m} \coeffA{m}{d} \binom{d}{2r+1} \frac{(-1)^{d-1}}{d-r}
    \bernoulli{2d-2r}
\end{equation*}
Finally, the coefficient $\coeffA{m}{r}$ is defined recursively as
\begin{equation}
    \label{eq:definition_coefficient_a}
    \coeffA{m}{r} \colonequals
    \begin{cases}
    (2r+1)
        \binom{2r}{r}, & \text{if } r=m; \\
        (2r+1) \binom{2r}{r} \sum_{d \geq 2r+1}^{m} \coeffA{m}{d} \binom{d}{2r+1} \frac{(-1)^{d-1}}{d-r}
        \bernoulli{2d-2r}, & \text{if } 0 \leq r<m; \\
        0, & \text{if } r<0 \text{ or } r>m,
    \end{cases}
\end{equation}
where $\bernoulli{t}$ are Bernoulli numbers~\cite{bateman1953higher}.
It is assumed that $\bernoulli{1}=\frac{1}{2}$.
For example,
\begin{table}[H]
    \begin{center}
        \setlength\extrarowheight{-6pt}
        \begin{tabular}{c|cccccccc}
            $m/r$ & 0 & 1       & 2      & 3      & 4   & 5    & 6     & 7 \\ [3px]
            \hline
            0     & 1 &         &        &        &     &      &       &       \\
            1     & 1 & 6       &        &        &     &      &       &       \\
            2     & 1 & 0       & 30     &        &     &      &       &       \\
            3     & 1 & -14     & 0      & 140    &     &      &       &       \\
            4     & 1 & -120    & 0      & 0      & 630 &      &       &       \\
            5     & 1 & -1386   & 660    & 0      & 0   & 2772 &       &       \\
            6     & 1 & -21840  & 18018  & 0      & 0   & 0    & 12012 &       \\
            7     & 1 & -450054 & 491400 & -60060 & 0   & 0    & 0     & 51480
        \end{tabular}
    \end{center}
    \caption{Coefficients $\coeffA{m}{r}$.}
    \label{tab:table_of_coefficients_a}
\end{table}
The coefficients $\coeffA{m}{r}$ are also registered in the OEIS~\cite{kolosov2018numerator,kolosov2018denominator}.
It is also interesting to notice that row sums of the $\coeffA{m}{r}$ give powers of $2$
\begin{equation*}
    \sum_{r=0}^{m} \coeffA{m}{r} = 2^{2m+1}
\end{equation*}