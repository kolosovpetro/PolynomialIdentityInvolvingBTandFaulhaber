\documentclass[12pt,letterpaper,oneside,reqno]{amsart}
\usepackage{amsfonts}
\usepackage{amsmath}
\usepackage{amssymb}
\usepackage{amsthm}
\usepackage{float}
\usepackage{mathrsfs}
\usepackage{colonequals}
\usepackage[font=small,labelfont=bf]{caption}
\usepackage[left=1in,right=1in,bottom=1in,top=1in]{geometry}
\usepackage[pdfpagelabels,hyperindex,colorlinks=true,linkcolor=blue,urlcolor=magenta,citecolor=green]{hyperref}
\usepackage{graphicx}
\linespread{1.7}
\usepackage{array}
\emergencystretch=1em

\newcommand \anglePower [2]{\langle #1 \rangle \sp{#2}}
\newcommand \bernoulli [2][B] {{#1}\sb{#2}}
\newcommand \curvePower [2]{\{#1\}\sp{#2}}
\newcommand \coeffA [3][A] {{\mathbf{#1}} \sb{#2,#3}}
\newcommand \polynomialP [4][P]{{\mathbf{#1}}\sp{#2} \sb{#3}(#4)}

% ordinary derivatives
\newcommand \derivative [2] {\frac{d}{d #2} #1}                              % 1 - function; 2 - variable;
\newcommand \pderivative [2] {\frac{\partial #1}{\partial #2}}               % 1 - function; 2 - variable;
\newcommand \qderivative [1] {D_{q} #1}                                      % 1 - function
\newcommand \nqderivative [1] {D_{n,q} #1}                                   % 1 - function
\newcommand \qpowerDerivative [1] {\mathcal{D}_q #1}                         % 1 - function;
\newcommand \finiteDifference [1] {\Delta #1}                                % 1 - function;
\newcommand \pTsDerivative [2] {\frac{\partial #1}{\Delta #2}}               % 1 - function; 2 - variable;

% high order derivatives
\newcommand \derivativeHO [3] {\frac{d^{#3}}{d {#2}^{#3}} #1}                % 1 - function; 2 - variable; 3 - order
\newcommand \pderivativeHO [3]{\frac{\partial^{#3}}{\partial {#2}^{#3}} #1}
\newcommand \qderivativeHO [2] {D_{q}^{#2} #1}                               % 1 - function; 2 - order
\newcommand \qpowerDerivativeHO [2] {\mathcal{D}_{q}^{#2} #1}                % 1 - function; 2 - order
\newcommand \finiteDifferenceHO [2] {\Delta^{#2} #1}                         % 1 - function; 2 - order
\newcommand \pTsDerivativeHO [3] {\frac{\partial^{#3}}{\Delta {#2}^{#3}} #1} % 1 - function; 2 - variable;

\newtheorem{thm}{Theorem}[section]
\newtheorem{cor}[thm]{Corollary}
\newtheorem{lem}[thm]{Lemma}
\newtheorem{examp}[thm]{Example}
\newtheorem{conj}[thm]{Conjecture}

\numberwithin{equation}{section}

\title[Polynomial identity involving Binomial Theorem and Faulhaber's formula]
{Polynomial identity involving Binomial Theorem and Faulhaber's formula}
\author[Petro Kolosov]{Petro Kolosov}
\email{kolosovp94@gmail.com}
\keywords{
    Binomial theorem,
    Polynomial identities,
    Binomial coefficients,
    Bernoulli numbers,
    Pascal's triangle,
    Faulhaber's formula
}
\urladdr{https://kolosovpetro.github.io}
\subjclass[2010]{26E70, 05A30}
\date{\today}
\hypersetup{
    pdftitle={Derivation of A Coefficients},
    pdfsubject={
        Binomial theorem,
        Polynomial identities,
        Binomial coefficients,
        Bernoulli numbers,
        Pascal's triangle,
        Faulhaber's formula},
    pdfauthor={Petro Kolosov},
    pdfkeywords={
        Binomial theorem,
        Polynomial identities,
        Binomial coefficients,
        Bernoulli numbers,
        Pascal's triangle,
        Faulhaber's formula}
}
\begin{document}

    \maketitle

    \tableofcontents


    \section{Introduction} \label{sec:introduction}
    Considering the table of forward finite differences of the polynomial $n^3$
\begin{table}[H]
    \begin{center}
        \setlength\extrarowheight{-6pt}
        \begin{tabular}{c|cccc}
            $n$ & $n^3$ & $\Delta(n^3)$ & $\Delta^2(n^3)$ & $\Delta^3(n^3)$ \\
            \hline
            0   & 0     & 1             & 6               & 6               \\
            1   & 1     & 7             & 12              & 6               \\
            2   & 8     & 19            & 18              & 6               \\
            3   & 27    & 37            & 24              & 6               \\
            4   & 64    & 61            & 30              & 6               \\
            5   & 125   & 91            & 36              &                 \\
            6   & 216   & 127           &                 &                 \\
            7   & 343   &               &                 &
        \end{tabular}
    \end{center}
    \caption{Table of finite differences of the polynomial $n^3$.} \label{tab:table}
\end{table}
We can easily observe that finite differences
\footnote{One may assume that it is possible to reach the form $n^{2m+1} = \sum_{k=1}^{n} \mathbf{A}_{m,0} k^0 (n-k)^0 + \mathbf{A}_{m,1}(n-k)^1
+ \cdots + \mathbf{A}_{m,m} k^m (n-k)^m$ simply taking finite differences of the polynomial $n^{2m+1}$ up to order of $2m+1$ 
and interpolating it backwards similarly as shown in~\eqref{eq:cubes_interpolation}.
However, my observations do not provide any evidence of such assumption.
Interestingly enough is that we could have been arrived to the pure differential approach of the relation~\eqref{eq:odd_power_conjecture} then.}
of the polynomial $n^3$ may be expressed according
to the following relation, via rearrangement of the terms
\begin{align}
    \label{eq:cubes_interpolation}
    \begin{split}
        \Delta(0^3) &= 1+6 \cdot 0 \\
        \Delta(1^3) &= 1+6\cdot0+6\cdot1 \\
        \Delta(2^3) &= 1+6\cdot0+6\cdot1+6\cdot2 \\
        \Delta(3^3) &= 1+6\cdot0+6\cdot1+6\cdot2+6\cdot3 \\
        &\; \; \vdots \\
        \Delta(n^3) &= 1+6\cdot0+6\cdot1+6\cdot2+6\cdot3+\cdots+6\cdot n
    \end{split}
\end{align}
Furthermore, the polynomial $n^3$ is equivalent to
\begin{align*}
    n^3 &= [1+6\cdot0]+[1+6\cdot0+6\cdot1]+[1+6\cdot0+6\cdot1+6\cdot2]+\cdots \\
    &+[1+6\cdot0+6\cdot1+6\cdot2+\cdots+6\cdot(n-1)]
\end{align*}
Rearranging the above equation, we get
\[
    n^3 = n +(n-0) \cdot6 \cdot0 + (n-1)\cdot6\cdot1 + (n-2)\cdot6\cdot2 + \cdots+1\cdot6\cdot(n-1)
\]
Therefore, we can consider the polynomial $n^3$ as
\begin{equation}
    \label{eq:cube_identity}
    n^3 = \sum_{k=1}^{n} 6k(n-k) + 1
\end{equation}
Assume that equation~\eqref{eq:cube_identity} has the following implicit form
\begin{equation}
    \label{eq:pattern}
    n^3 = \sum_{k=1}^{n} \coeffA{1}{1} k^1(n-k)^1 + \coeffA{1}{0} k^0(n-k)^0,
\end{equation}
where $\coeffA{1}{1} = 6$ and $\coeffA{1}{0} = 1$, respectively.
Note that here the power of $3$ is actually defined by $2m+1$ where $m=1$.
So, is there a generalization of the relation~\eqref{eq:pattern} for all positive odd powers $2m+1, \; m=0,1,2,\dots$?
Therefore, let us propose a conjecture
\begin{conj}
    For every $n\geq 1, \; n,m\in\mathbb{N}$ there are coefficients $\coeffA{m}{0}, \coeffA{m}{1}, \ldots, \coeffA{m}{m}$ such that
    \begin{equation}
        \label{eq:odd_power_conjecture}
        n^{2m+1} = \sum_{k=1}^{n} \coeffA{m}{0} k^0 (n-k)^0 + \coeffA{m}{1} (n-k)^1
        + \cdots + \coeffA{m}{m} k^m (n-k)^m
    \end{equation}
\end{conj}

%
%
%    \section{Coefficients derivation}\label{sec:coefficients-derivation}
%    Consider the Faulhaber's formula
\begin{equation}
    \sum_{k=1}^{n} k^{p} = \frac{1}{p+1}\sum_{j=0}^{p} \binom{p+1}{j} \bernoulli{j} n^{p+1-j}\label{eq:equation3}
\end{equation}
it is very important to note that summation bound is $p$ while binomial coefficient upper bound is $p+1$.
It means that we cannot skip summation bounds unless we do some trick as
\begin{equation}
    \begin{split}
        \sum_{k=1}^{n} k^{p} &= \frac{1}{p+1}\sum_{j=0}^{p} \binom{p+1}{j} \bernoulli{j} n^{p+1-j} \\
        &= \left[ \frac{1}{p+1}\sum_{j=0}^{p+1} \binom{p+1}{j} \bernoulli{j} n^{p+1-j} \right] - \bernoulli{p+1} \\
        &= \left[ \frac{1}{p+1}\sum_{j} \binom{p+1}{j} \bernoulli{j} n^{p+1-j} \right] - \bernoulli{p+1} \\
    \end{split}\label{eq:faulhaber-formula}
\end{equation}
Using Faulhaber's formula
$\sum_{k=1}^{n} k^{p} = \left[ \frac{1}{p+1}\sum_{j} \binom{p+1}{j} \bernoulli{j} n^{p+1-j} \right] - \bernoulli{p+1}$
we get
\begin{equation}
    \begin{split}
        \sum_{k=1}^{n} k^{r} (n-k)^{r}
        &= \sum_{t=0}^{r} (-1)^t \binom{r}{t} n^{r-t} \sum_{k=1}^{n} k^{t+r} \\
        &= \sum_{t=0}^{r} (-1)^t \binom{r}{t} n^{r-t} \left[ \frac{1}{t+r+1} \sum_{j} \binom{t+r+1}{j} \bernoulli{j} n^{t+r+1-j} - \bernoulli{t+r+1} \right] \\
        &= \sum_{t=0}^{r} \binom{r}{t} \left[ \frac{(-1)^t}{t+r+1} \sum_{j} \binom{t+r+1}{j} \bernoulli{j} n^{2r+1-j} - \bernoulli{t+r+1} n^{r-t} \right] \\
        &= \sum_{t=0}^{r} \binom{r}{t} \frac{(-1)^t}{t+r+1} \sum_{j} \binom{t+r+1}{j} \bernoulli{j} n^{2r+1-j} - \sum_{t=0}^{r} \binom{r}{t} \frac{(-1)^t}{t+r+1} \bernoulli{t+r+1} n^{r-t} \\
        &= \sum_{j} \sum_{t} \binom{r}{t} \frac{(-1)^t}{t+r+1} \binom{t+r+1}{j} \bernoulli{j} n^{2r+1-j} - \sum_{t=0}^{r} \binom{r}{t} \frac{(-1)^t}{t+r+1} \bernoulli{t+r+1} n^{r-t} \\
        &= \sum_{j} \bernoulli{j} n^{2r+1-j} \sum_{t} \binom{r}{t} \frac{(-1)^t}{t+r+1} \binom{t+r+1}{j} - \sum_{t=0}^{r} \binom{r}{t} \frac{(-1)^t}{t+r+1} \bernoulli{t+r+1} n^{r-t}
    \end{split}\label{eq:equation4}
\end{equation}
Now, we notice that
\begin{equation}
    \sum_{t} \binom{r}{t} \frac{(-1)^t}{r+t+1} \binom{r+t+1}{j}
    =\begin{cases}
         \frac{1}{(2r+1) \binom{2r}r}, & \text{if } j=0;\\
         \frac{(-1)^r}{j} \binom{r}{2r-j+1}, & \text{if } j>0.
    \end{cases}\label{eq:combinatorial-identity}
\end{equation}
In particular, the last sum is zero for $0< t \leq j$.
So taking $j=0$ we have
\begin{equation}
    \begin{split}
        \sum_{k=1}^{n} k^{r} (n-k)^{r}
        &= \frac{1}{(2r+1) \binom{2r}r} n^{2r+1} + \left[ \sum_{j \geq 1} \bernoulli{j} n^{2r+1-j} \sum_{t} \binom{r}{t} \frac{(-1)^t}{t+r+1} \binom{t+r+1}{j} \right] \\
        &- \left[ \sum_{t=0}^{r} \binom{r}{t} \frac{(-1)^t}{t+r+1} \bernoulli{t+r+1} n^{r-t} \right]
    \end{split}\label{eq:equation5}
\end{equation}
Now let's simplify the double summation
\begin{equation}
    \begin{split}
        \sum_{k=1}^{n} k^{r} (n-k)^{r}
        &= \frac{1}{(2r+1) \binom{2r}r} n^{2r+1}
        + \underbrace{\left[ \sum_{j \geq 1} \frac{(-1)^r}{j} \binom{r}{2r-j+1} \bernoulli{j} n^{2r+1-j} \right]}_{(\star)} \\
        &- \underbrace{\left[ \sum_{t=0}^{r} \binom{r}{t} \frac{(-1)^t}{t+r+1} \bernoulli{t+r+1} n^{r-t} \right]}_{(\diamond)}
    \end{split}\label{eq:equation6}
\end{equation}
Hence, introducing $\ell=2r-j+1$ to $(\star)$ and $\ell=r-t$ to $(\diamond)$, we get
\begin{equation}
    \begin{split}
        \sum_{k=1}^{n} k^{r} (n-k)^{r}
        &= \frac{1}{(2r+1) \binom{2r}r} n^{2r+1}
        + \left[ \sum_{\ell} \frac{(-1)^r}{2r+1-\ell} \binom{r}{\ell} \bernoulli{2r+1-\ell} n^{\ell} \right] \\
        &- \left[ \sum_{\ell} \binom{r}{\ell} \frac{(-1)^{r-\ell}}{2r+1-\ell} \bernoulli{2r+1-\ell} n^{\ell} \right]\\
        &= \frac{1}{(2r+1) \binom{2r}r} n^{2r+1} + 2 \sum_{\mathrm{odd \; \ell}} \frac{(-1)^r}{2r+1-\ell} \binom{r}{\ell} \bernoulli{2r+1-\ell} n^{\ell}
    \end{split}\label{eq:polynomial-sum-1}
\end{equation}
Using the definition of $\coeffA{m}{r}$, we obtain the following identity for polynomials in $n$
\begin{equation}
    \label{eq:equation}
    \sum_{r} \coeffA{m}{r} \frac{1}{(2r+1) \binom{2r}r} n^{2r+1}
    + 2 \sum_{r} \coeffA{m}{r} \sum_{\mathrm{odd \; \ell}} \frac{(-1)^r}{2r+1-\ell} \binom{r}{\ell} \bernoulli{2r+1-\ell} n^{\ell}
    \equiv n^{2m+1}
\end{equation}
Replacing odd $\ell$ by $d$ we get
\begin{equation}
    \begin{split}
        &\sum_{r} \coeffA{m}{r} \frac{1}{(2r+1) \binom{2r}r} n^{2r+1}
        + 2 \sum_{r} \coeffA{m}{r} \sum_{d} \frac{(-1)^r}{2r-2d} \binom{r}{2d+1} \bernoulli{2r-2d} n^{2d+1}
        \equiv n^{2m+1} \\
        &\sum_{r} \coeffA{m}{r} \left[ \frac{1}{(2r+1) \binom{2r}r} n^{2r+1} \right]
        + 2 \sum_{r} \coeffA{m}{r} \left[ \sum_{d} \frac{(-1)^r}{2r-2d} \binom{r}{2d+1} \bernoulli{2r-2d} n^{2d+1} \right]
        - n^{2m+1} = 0 \\
        &\sum_{r=0}^{m} \coeffA{m}{r} \left[ \frac{1}{(2r+1) \binom{2r}r} n^{2r+1} \right]
        + 2 \sum_{r=0}^{m} \coeffA{m}{r} \left[ \sum_{d=0}^{(r-1)/2} \frac{(-1)^r}{2r-2d} \binom{r}{2d+1} \bernoulli{2r-2d} n^{2d+1} \right]
        - n^{2m+1} = 0 \\
    \end{split}\label{eq:equation7}
\end{equation}
Taking the coefficient of $n^{2m+1}$ in~\eqref{eq:equation7}, we get
\begin{equation}
    \coeffA{m}{m} = (2m+1)\binom{2m}{m}\label{eq:equation8}
\end{equation}
and taking the coefficient of $x^{2d+1}$ for an integer $d$ in the range $m/2 \leq d < m$, we get
\begin{equation}
    \coeffA{m}{d} = 0\label{eq:equation9}
\end{equation}
Taking the coefficient of $n^{2d+1}$ for $d$ in the range $m/4 \leq d < m/2$ we get
\begin{equation}
    \coeffA{m}{d} \frac{1}{(2d+1) \binom{2d}{d}}
    +2 \underbrace{(2m+1) \binom{2m}{m}}_{\coeffA{m}{m}} \binom{m}{2d+1} \frac{(-1)^m}{2m-2d} \bernoulli{2m-2d} = 0,\label{eq:equation10}
\end{equation}
i.e
\begin{equation}
    \coeffA{m}{d} = (-1)^{m-1} \frac{(2m+1)!}{d!d!m!(m-2d-1)!} \frac{1}{m-d} \bernoulli{2m-2d}\label{eq:equation11}
\end{equation}
Continue similarly we can express $\coeffA{m}{r}$ for each integer $r$ in range $m/2^{s+1}\leq r < m/2^s$
(iterating consecutively $s=1,2,\ldots$) via previously determined values of $\coeffA{m}{d}$ as follows
\begin{equation}
    \coeffA{m}{r} =
    (2r+1) \binom{2r}{r} \sum_{d=2r+1}^{m} \coeffA{m}{d} \binom{d}{2r+1} \frac{(-1)^{d-1}}{d-r}
    \bernoulli{2d-2r}\label{eq:equation12}
\end{equation}
Finally, the coefficient $\coeffA{m}{r}$ is defined recursively as
\begin{equation}
    \label{eq:def_coeff_a}
    \coeffA{m}{r} \colonequals
    \begin{cases}
    (2r+1)
        \binom{2r}{r}, & \text{if } r=m; \\
        (2r+1) \binom{2r}{r} \sum_{d=2r+1}^{m} \coeffA{m}{d} \binom{d}{2r+1} \frac{(-1)^{d-1}}{d-r}
        \bernoulli{2d-2r}, & \text{if } 0 \leq r<m; \\
        0, & \text{if } r<0 \text{ or } r>m,
    \end{cases}
\end{equation}
where $\bernoulli{t}$ are Bernoulli numbers.
It is assumed that $\bernoulli{1}=\frac{1}{2}$.
\begin{examp}
    Example for $\coeffA{m}{r}$ for $m=2$.
    First we get $\coeffA{2}{2}$
    \begin{equation*}
        \coeffA{m}{m} = 5\binom{4}{2}=30
    \end{equation*}
    Then $\coeffA{2}{1} = 0$ because $\coeffA{m}{d}$ is zero in the range $m/2 \leq d < m$ means that zero for $d$ in
    $1 \leq d < 2$.
    Finally, the $\coeffA{2}{0}$ is
    \begin{equation*}
        \coeffA{2}{0} = 1 \binom{0}{0} \sum_{d \geq 1} \coeffA{2}{d} \binom{d}{1} \frac{(-1)^{d-1}}{d} \bernoulli{2d}
        = \coeffA{2}{2} 2 \frac{-1}{2} \bernoulli{4} = 1.
    \end{equation*}
\end{examp}
\begin{examp}
    Example for $\coeffA{m}{r}$ for $m=3$.
    First we get $\coeffA{3}{3}$
    \begin{equation*}
        \coeffA{m}{m} = 7 \binom{6}{3}= 140
    \end{equation*}
    Then $\coeffA{3}{2} = 0$ because $\coeffA{m}{d}$ is zero in the range $m/2 \leq d < m$ means that zero for $d$ in
    $2 \leq d < 3$.
    The $\coeffA{3}{1}$ coefficient is non-zero and calculated as
    \begin{equation*}
        \begin{split}
            \coeffA{3}{1}
            &= 3 \binom{2}{1} \sum_{d \geq 3} \coeffA{3}{d} \binom{d}{3} \frac{(-1)^{d-1}}{d} \bernoulli{2d-2} \\
            &= 3 \binom{2}{1} \coeffA{3}{3} \binom{3}{3} \frac{(-1)^2}{2} \bernoulli{6}
            = 3 \cdot 140 \cdot (-\frac{1}{30}) = -14
        \end{split}
    \end{equation*}
    Finally $\coeffA{3}{0}$ coefficient is
    \begin{equation*}
        \begin{split}
            \coeffA{3}{0}
            &= 1 \binom{0}{0} \sum_{d \geq 1} \coeffA{3}{d} \binom{d}{1} \frac{(-1)^{d-1}}{d} \bernoulli{2d}
            = \sum_{d \geq 1} \coeffA{3}{d} \binom{d}{1} \frac{(-1)^{d-1}}{d} \bernoulli{2d} \\
            & = \coeffA{3}{1} \binom{1}{1} \frac{(-1)^{1-1}}{1} \bernoulli{2}
            + \coeffA{3}{2} \binom{2}{1} \frac{(-1)^{2-1}}{2} \bernoulli{4}
            + \coeffA{3}{3} \binom{3}{1} \frac{(-1)^{3-1}}{3} \bernoulli{6}
        \end{split}
    \end{equation*}
\end{examp}
%
%
%    \section{Example 1}\label{sec:example-1}
%    For $m=1$ we have an identity

\begin{equation*}
    \coeffA{m}{0} n + \coeffA{m}{1} \left[ \frac{1}{6} (-n + n^3) \right] - n^3 = 0
\end{equation*}
Multiplying by 6 both parts we get
\begin{equation*}
    6 \coeffA{m}{0} n + \coeffA{m}{1} (-n + n^3)  - 6 n^3 = 0
\end{equation*}

Opening brackets gives
\begin{equation*}
    6 \coeffA{m}{0} n - \coeffA{m}{1} n + \coeffA{m}{1} n^3  - 6 n^3 = 0
\end{equation*}

Arranging terms we get

\begin{equation*}
    n (6\coeffA{m}{0} - \coeffA{m}{1})+ n^3 (\coeffA{m}{1} - 6)  = 0, \quad n \geq 1
\end{equation*}

Hence

\begin{equation*}
    \begin{cases}
        6\coeffA{m}{0} - \coeffA{m}{1} = 0 \\
        \coeffA{m}{1} - 6 = 0
    \end{cases}
\end{equation*}

So that $\coeffA{m}{1}=6 \quad \coeffA{m}{1} = 1$
%
%
%    \section{Example 2}\label{sec:example-2}
%    For $m=2$ we have an identity

\begin{equation*}
    \coeffA{m}{0} n
    + \coeffA{m}{1} \left[ \frac{1}{6} (-n + n^3) \right]
    + \coeffA{m}{2} \left[ \frac{1}{30} (-n + n^5) \right] - n^5 = 0
\end{equation*}
Multiplying by 30 both parts we get
\begin{equation*}
    30 \coeffA{m}{0} n
    + 5 \coeffA{m}{1}  (-n + n^3)
    + \coeffA{m}{2} (-n + n^5) - 30 n^5 = 0
\end{equation*}

Opening brackets gives
\begin{equation*}
    30 \coeffA{m}{0} n -5 \coeffA{m}{1} n + 5 \coeffA{m}{1} n^3 -\coeffA{m}{2} n + n^5 \coeffA{m}{2} n^5 - 30 n^5 = 0
\end{equation*}

Arranging terms we get

\begin{equation*}
    n (30 \coeffA{m}{0} n - 5 \coeffA{m}{1} -\coeffA{m}{2}) + 5 \coeffA{m}{1} n^3 + n^5 (\coeffA{m}{2} n^5 - 30) = 0, \quad n \geq 1
\end{equation*}

Hence

\begin{equation*}
    \begin{cases}
        \coeffA{m}{1} = 0 \\
        \coeffA{m}{2} n^5 - 30 = 0 \\
        30 \coeffA{m}{0} n - 5 \coeffA{m}{1} -\coeffA{m}{2} = 0
    \end{cases}
\end{equation*}

So that $\coeffA{m}{1}=6 \quad \coeffA{m}{1} = 1$
%
%
%    \section{Example 3}\label{sec:example-3}
%    \input{sections/example3}
%
%
%    \section{Example 4}\label{sec:example-4}
%    \input{sections/example4}
%
%
%    \section{Example 5}\label{sec:example-5}
%    \input{sections/example5}

%    \section{Examples}\label{sec:examples}
%    Let be an example for $m=2$ of
\begin{equation*}
    \sum_{r=0}^{m} \coeffA{m}{r} \left[ \frac{1}{(2r+1) \binom{2r}r} n^{2r+1} \right]
    + 2 \sum_{r=0}^{m} \coeffA{m}{r} \left[ \sum_{d=0}^{(r-1)/2} \frac{(-1)^r}{2r-2d} \binom{r}{2d+1} \bernoulli{2r-2d} n^{2d+1} \right]
    - n^{2m+1} = 0
\end{equation*}
So that
\begin{equation*}
    \sum_{r=0}^{2} \coeffA{2}{r} \left[ \frac{1}{(2r+1) \binom{2r}r} n^{2r+1} \right]
    + 2 \sum_{r=0}^{2} \coeffA{2}{r} \left[ \sum_{d=0}^{(r-1)/2} \frac{(-1)^r}{2r-2d} \binom{r}{2d+1} \bernoulli{2r-2d} n^{2d+1} \right]
    - n^{5} = 0
\end{equation*}
The sum $\sum_{r=0}^{2} \coeffA{2}{r} \left[ \frac{1}{(2r+1) \binom{2r}r} n^{2r+1} \right]$ in explicit form is
\begin{equation*}
    \sum_{r=0}^{2} \coeffA{2}{r} \left[ \frac{1}{(2r+1) \binom{2r}r} n^{2r+1} \right] =
    \coeffA{2}{0} n
    + \coeffA{2}{1} \left[ \frac{1}{3 \binom{2}{1}} n^{3} \right]
    + \coeffA{2}{2} \left[ \frac{1}{5 \binom{4}{2}} n^{5} \right]
\end{equation*}
Also the sum
$\sum_{r=0}^{2} \coeffA{2}{r} \left[ \sum_{d=0}^{(r-1)/2} \frac{(-1)^r}{2r-2d} \binom{r}{2d+1} \bernoulli{2r-2d} n^{2d+1} \right]$
is
\begin{equation*}
    \begin{split}
        &\sum_{r=0}^{2} \coeffA{2}{r} \left[ \sum_{d=0}^{(r-1)/2} \frac{(-1)^r}{2r-2d} \binom{r}{2d+1} \bernoulli{2r-2d} n^{2d+1} \right] = \\
        & + \coeffA{2}{0} \left[ \sum_{d=0}^{-1} \frac{1}{-2d} \binom{0}{2d+1} \bernoulli{-2d} n^{2d+1} \right] \\
        & - \coeffA{2}{1} \left[ \sum_{d=0}^{0} \frac{1}{2-2d} \binom{1}{2d+1} \bernoulli{2-2d} n^{2d+1} \right] \\
        & + \coeffA{2}{2} \left[ \sum_{d=0}^{0} \frac{1}{4-2d} \binom{2}{2d+1} \bernoulli{4-2d} n^{2d+1} \right] \\
        & = \coeffA{2}{0} \cdot 0
        - \coeffA{2}{1} \left[ \frac{1}{2} \binom{1}{1} \bernoulli{2} n^{1} \right]
        + \coeffA{2}{2} \left[ \frac{1}{4} \binom{2}{1} \bernoulli{4} n^{1} \right] \\
        & = \coeffA{2}{0} \cdot 0
        - \coeffA{2}{1} \left[ \frac{1}{12} n \right]
        - \coeffA{2}{2} \left[ \frac{1}{60} n \right] \\
        & = - \coeffA{2}{1} \frac{1}{12} n
        - \coeffA{2}{2}  \frac{1}{60} n
    \end{split}
\end{equation*}



%    \section{Faulhaber's formulae}\label{sec:faulhaber-formulae}
%    \begin{equation*}
    \begin{split}
        \sum_{k=1}^{n} k^{1} &= \frac{1}{2} (-n + n^2) \\
        \sum_{k=1}^{n} k^{2} &= \frac{1}{6} (n - 3 n^2 + 2 n^3) \\
        \sum_{k=1}^{n} k^{3} &= \frac{1}{4} (n^2 - 2 n^3 + n^4) \\
        \sum_{k=1}^{n} k^{4} &= \frac{1}{30} (-n + 10 n^3 - 15 n^4 + 6 n^5) \\
        \sum_{k=1}^{n} k^{5} &= \frac{1}{12} (-n^2 + 5 n^4 - 6 n^5 + 2 n^6) \\
        \sum_{k=1}^{n} k^{6} &= \frac{1}{42} (n - 7 n^3 + 21 n^5 - 21 n^6 + 6 n^7) \\
        \sum_{k=1}^{n} k^{7} &= \frac{1}{24} (2 n^2 - 7 n^4 + 14 n^6 - 12 n^7 + 3 n^8) \\
        \sum_{k=1}^{n} k^{8} &= \frac{1}{90} (-3 n + 20 n^3 - 42 n^5 + 60 n^7 - 45 n^8 + 10 n^9) \\
    \end{split}
\end{equation*}
%
%
%    \section{Binomial power sums formulae}\label{sec:binomial-power-sums-formulae}
%    \begin{equation*}
    \begin{split}
        \sum_{t=0}^{1} (-1)^t \binom{1}{t} n^{1-t} \sum_{k=1}^{n} k^{t+1} =& \frac{1}{6} (-n + n^3) \\
        \sum_{t=0}^{2} (-1)^t \binom{2}{t} n^{2-t} \sum_{k=1}^{n} k^{t+2} =& \frac{1}{30} (-n + n^5) \\
        \sum_{t=0}^{3} (-1)^t \binom{3}{t} n^{3-t} \sum_{k=1}^{n} k^{t+3} =& \frac{1}{420} (-10 n + 7 n^3 + 3 n^7) \\
        \sum_{t=0}^{4} (-1)^t \binom{4}{t} n^{4-t} \sum_{k=1}^{n} k^{t+4} =& \frac{1}{630} (-21 n + 20 n^3 + n^9) \\
        \sum_{t=0}^{5} (-1)^t \binom{5}{t} n^{5-t} \sum_{k=1}^{n} k^{t+5} =& \frac{1}{2772} (-210 n + 231 n^3 - 22 n^5 + n^{11}) \\
        \sum_{t=0}^{6} (-1)^t \binom{6}{t} n^{6-t} \sum_{k=1}^{n} k^{t+6} =& \frac{1}{60060} (-15202 n + 18200 n^3 - 3003 n^5 + 5 n^{13}) \\
        \sum_{t=0}^{7} (-1)^t \binom{7}{t} n^{7-t} \sum_{k=1}^{n} k^{t+7} =& \frac{1}{51480} (-60060 n + 76010 n^3 - 16380 n^5 + 429 n^7 + n^{15}) \\
        \sum_{t=0}^{8} (-1)^t \binom{8}{t} n^{8-t} \sum_{k=1}^{n} k^{t+8} =& \frac{1}{218790} (-1551693 n + 2042040 n^3 - 516868 n^5 + 26520 n^7 + n^{17})
    \end{split}
\end{equation*}
%
%
%    \section{Questions}\label{sec:questions}
%    \begin{enumerate}
    \item Any proof or reference to the relation~\eqref{eq:combinatorial-identity}?
    \item What is the motivation to use~\eqref{eq:faulhaber-formula} version of Faulhaber's formula?
    \item Why is there twice odd $\ell$ in~\eqref{eq:polynomial-sum-1}?
\end{enumerate}


%    \section{Conclusions}\label{sec:conclusions}
%    In this manuscript, we have shown that for every $n\geq 1, \; n,m\in\mathbb{N}$
there are coefficients $\mathbf{A}_{m,0}, \mathbf{A}_{m,1}, \ldots, \mathbf{A}_{m,m}$ such that
the polynomial identity holds
\[
    n^{2m+1} = \sum_{k=1}^{n} \mathbf{A}_{m,0} k^0 (n-k)^0 + \mathbf{A}_{m,1}(n-k)^1
    + \cdots + \mathbf{A}_{m,m} k^m (n-k)^m
\]
In particular, the coefficients $\coeffA{m}{r}$ can be evaluated in both ways,
by constructing and solving a certain system of linear equations or by deriving a recurrence relation;
all these approaches are examined providing examples
in the sections~\eqref{sec:approach-via-system-of-linear-equations} and~\eqref{sec:finding-a-recurrence-relation}.
Moreover, to validate the results, supplementary Mathematica programs are available at~\cite{kolosov2023github}.

%
%    \bibliographystyle{unsrt}
%    \bibliography{CoefficientsADerivationReferences}

\end{document}
