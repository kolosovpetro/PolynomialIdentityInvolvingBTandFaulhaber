Back then in 2016 I was playing with numbers and have noticed the pattern in terms of finite differences
of cubes $n^3$.
Considering the table of forward finite differences of the polynomial $n^3$
\begin{table}[H]
    \begin{center}
        \setlength\extrarowheight{-6pt}
        \begin{tabular}{c|cccc}
            $n$ & $n^3$ & $\Delta(n^3)$ & $\Delta^2(n^3)$ & $\Delta^3(n^3)$ \\
            \hline
            0   & 0     & 1             & 6               & 6               \\
            1   & 1     & 7             & 12              & 6               \\
            2   & 8     & 19            & 18              & 6               \\
            3   & 27    & 37            & 24              & 6               \\
            4   & 64    & 61            & 30              & 6               \\
            5   & 125   & 91            & 36              &                 \\
            6   & 216   & 127           &                 &                 \\
            7   & 343   &               &                 &
        \end{tabular}
    \end{center}
    \caption{Table of finite differences of the polynomial $n^3$} \label{tab:table}
\end{table}
We can observe easily that finite differences of the polynomial $n^3$ may be expressed according
to the following relation, via rearrangement of the terms
\begin{align*}
    \Delta(0^3) &= 1+6 \cdot 0 \\
    \Delta(1^3) &= 1+6\cdot0+6\cdot1 \\
    \Delta(2^3) &= 1+6\cdot0+6\cdot1+6\cdot2 \\
    \Delta(3^3) &= 1+6\cdot0+6\cdot1+6\cdot2+6\cdot3 \\
    &\; \; \vdots \\
    \Delta(n^3) &= 1+6\cdot0+6\cdot1+6\cdot2+6\cdot3+\cdots+6\cdot n
\end{align*}
Furthermore, the polynomial $n^3$ is identical to
\begin{align*}
    n^3 &= [1+6\cdot0]+[1+6\cdot0+6\cdot1]+[1+6\cdot0+6\cdot1+6\cdot2]+\cdots \\
    &+[1+6\cdot0+6\cdot1+6\cdot2+\cdots+6\cdot(n-1)]
\end{align*}
Rearranging above equation we get
\[
    n^3 = n +(n-0) \cdot6 \cdot0 + (n-1)\cdot6\cdot1 + (n-2)\cdot6\cdot2 + \cdots+1\cdot6\cdot(n-1)
\]
Therefore, we can consider $n^3$ as
\begin{equation}
    \label{eq:cube_identity}
    n^3 = \sum_{k=1}^{n} 6k(n-k) + 1
\end{equation}
Assume that equation~\eqref{eq:cube_identity} has an implicit form as follows
\begin{equation}
    \label{eq:pattern}
    n^3 = \sum_{k=1}^{n} \coeffA{1}{1} k^1(n-k)^1 + \coeffA{1}{0} k^0(n-k)^0,
\end{equation}
where $\coeffA{1}{1} = 6$ and $\coeffA{1}{0} = 1$, respectively.
So could the relation~\eqref{eq:pattern} be generalised for all positive odd powers?
Therefore, let be a conjecture
\begin{conj}
    For every $n\geq 1, \; n,m\in\mathbb{N}$ there are coefficients $\coeffA{m}{0}, \coeffA{m}{1}, \ldots, \coeffA{m}{m}$ such that
    \begin{equation*}
        \label{conjecture}
        n^{2m+1} = \sum_{k=1}^{n} \coeffA{m}{0} k^0 (n-k)^0 + \coeffA{m}{1} (n-k)^1
        + \cdots + \coeffA{m}{m} k^m (n-k)^m
    \end{equation*}
\end{conj}