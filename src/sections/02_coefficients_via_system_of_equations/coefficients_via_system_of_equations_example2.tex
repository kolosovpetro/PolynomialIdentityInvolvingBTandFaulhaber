\begin{examp}
    Let be $m=2$ so that we have the following relation defined by~\eqref{eq:arbitrary-relation}
    \begin{equation*}
        \coeffA{m}{0} n
        + \coeffA{m}{1} \left[ \frac{1}{6} (-n + n^3) \right]
        + \coeffA{m}{2} \left[ \frac{1}{30} (-n + n^5) \right] - n^5 = 0
    \end{equation*}
    Multiplying by $30$ right-hand side and left-hand side, we get
    \begin{equation*}
        30 \coeffA{2}{0} n + 5 \coeffA{2}{1} (-n + n^3) + \coeffA{2}{2} (-n + n^5) - 30n^5 = 0
    \end{equation*}
    Opening brackets and rearranging the terms gives
    \begin{equation*}
        30 \coeffA{2}{0} - 5 \coeffA{2}{1} n + 5 \coeffA{2}{1} n^3 - \coeffA{2}{2} n + \coeffA{2}{2} n^5 - 30n^5 = 0
    \end{equation*}
    Combining the common terms yields
    \begin{equation*}
        n (30 \coeffA{2}{0} - 5 \coeffA{2}{1} - \coeffA{2}{2}) + 5 \coeffA{2}{1} n^3 + n^5 (\coeffA{2}{2} - 30) = 0
    \end{equation*}
    Therefore, the system of linear equations follows
    \begin{equation*}
        \begin{cases}
            30 \coeffA{2}{0} - 5 \coeffA{2}{1} - \coeffA{2}{2} = 0 \\
            \coeffA{2}{1} = 0 \\
            \coeffA{2}{2} - 30 = 0
        \end{cases}
    \end{equation*}
    Solving it, we get
    \begin{equation*}
        \begin{cases}
            \coeffA{2}{2} = 30 \\
            \coeffA{2}{1} = 0 \\
            \coeffA{2}{0} = 1
        \end{cases}
    \end{equation*}
    So that odd-power identity~\eqref{eq:odd-power-identity} holds
    \begin{equation*}
        n^5 = \sum_{k=1}^{n} 30k^2(n-k)^2 + 1
    \end{equation*}
    It is also clearly seen
    why the above identity is true evaluating the terms $30k^2(n-k)^2 + 1$ over $0 \leq k \leq n$ as
    the following table shows
    \begin{table}[H]
        \setlength\extrarowheight{-6pt}
        \begin{tabular}{c|cccccccc}
            $n/k$ & 0 & 1    & 2    & 3    & 4    & 5    & 6    & 7 \\
            \hline
            0     & 1 &      &      &      &      &      &      &   \\
            1     & 1 & 1    &      &      &      &      &      &   \\
            2     & 1 & 31   & 1    &      &      &      &      &   \\
            3     & 1 & 121  & 121  & 1    &      &      &      &   \\
            4     & 1 & 271  & 481  & 271  & 1    &      &      &   \\
            5     & 1 & 481  & 1081 & 1081 & 481  & 1    &      &   \\
            6     & 1 & 751  & 1921 & 2431 & 1921 & 751  & 1    &   \\
            7     & 1 & 1081 & 3001 & 4321 & 4321 & 3001 & 1081 & 1
        \end{tabular}
        \caption{Values of $30k^2(n-k)^2 + 1$.
        See the OEIS entry \href{https://oeis.org/A300656}{\texttt{A300656}}~\cite{kolosov2018fifth}}
        \label{tab:row-sums-gives-fifth-power}
    \end{table}
\end{examp}
