\begin{example}
    Let be $m=4$ so that we have the following relation defined by~\eqref{eq:arbitrary-relation}
    \begin{equation*}
        \begin{split}
            \coeffA{m}{0} n
            &+ \coeffA{m}{1} \left[ \frac{1}{6} (-n + n^3) \right]
            + \coeffA{m}{2} \left[ \frac{1}{30} (-n + n^5) \right] \\
            &+ \coeffA{m}{3} \left[ \frac{1}{420} (-10 n + 7 n^3 + 3 n^7) \right] \\
            &+ \coeffA{m}{4} \left[ \frac{1}{630} (-21 n + 20 n^3 + n^9) \right] - n^9 = 0
        \end{split}
    \end{equation*}
    Multiplying by $630$ right-hand side and left-hand side, we get
    \begin{equation*}
        \begin{split}
            630 \coeffA{4}{0} n
            &+ 105 \coeffA{4}{1} (-n + n^3) + 21 \coeffA{4}{2} (-n + n^5) \\
            &+ \frac{3}{2} \coeffA{4}{3} (-10 n + 7 n^3 + 3 n^7) \\
            &+ \coeffA{4}{4} (-21 n + 20 n^3 + n^9) - 630 n^9 = 0
        \end{split}
    \end{equation*}
    Opening brackets and rearranging the terms gives
    \begin{equation*}
        \begin{split}
            630 \coeffA{4}{0} n
            &- 105 \coeffA{4}{1} n + 105 \coeffA{4}{1} n ^3 - 21 \coeffA{4}{2} n + 21 \coeffA{4}{2} n^5 \\
            &- \frac{3}{2} \coeffA{4}{3} \cdot 10n + \frac{3}{2} \coeffA{4}{3} \cdot 7n^3 + \frac{3}{2} \coeffA{4}{3} \cdot 3n^7 \\
            &-21 \coeffA{4}{4} n + 20 \coeffA{4}{4} n^3 + \coeffA{4}{4} n ^9 - 630 n^9 = 0
        \end{split}
    \end{equation*}
    Combining the common terms yields
    \begin{equation*}
        \begin{split}
            &n (630 \coeffA{4}{0} - 105 \coeffA{4}{1} - 21 \coeffA{4}{2} - 15 \coeffA{4}{3} - 21 \coeffA{4}{4})  \\
            &+ n^3 \left( 105 \coeffA{4}{1} + \frac{21}{2} \coeffA{4}{3} + 20 \coeffA{4}{4} \right) + n^5 (21 \coeffA{4}{2}) \\
            &+ n^7 \left( \frac{9}{2} \coeffA{4}{3} \right) + n^9 (\coeffA{4}{4} - 630) = 0
        \end{split}
    \end{equation*}
    Therefore, the system of linear equations follows
    \begin{equation*}
        \begin{cases}
            630 \coeffA{4}{0} - 105 \coeffA{4}{1} - 21 \coeffA{4}{2} - 15 \coeffA{4}{3} - 21 \coeffA{4}{4} = 0 \\
            105 \coeffA{4}{1} + \frac{21}{2} \coeffA{4}{3} + 20 \coeffA{4}{4} = 0 \\
            \coeffA{4}{2} = 0 \\
            \coeffA{4}{3} = 0 \\
            \coeffA{4}{4} - 630 = 0
        \end{cases}
    \end{equation*}
    Solving it, we get
    \begin{equation*}
        \begin{cases}
            \coeffA{4}{4} = 630 \\
            \coeffA{4}{3} = 0 \\
            \coeffA{4}{2} = 0 \\
            \coeffA{4}{1} = - \frac{20}{105} \coeffA{4}{4} = -120 \\
            \coeffA{4}{0} = \frac{105 \coeffA{4}{1} + 21 \coeffA{4}{4}}{630} = 1
        \end{cases}
    \end{equation*}
    So that odd-power identity~\eqref{eq:odd-power-identity} holds
    \begin{equation*}
        n^9 = \sum_{k=1}^{n} 630 k^4(n-k)^4 - 120k(n-k) + 1
    \end{equation*}
\end{example}
