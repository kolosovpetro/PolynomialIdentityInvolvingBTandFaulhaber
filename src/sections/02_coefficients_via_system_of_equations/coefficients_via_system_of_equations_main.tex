\documentclass[12pt,letterpaper,oneside,reqno]{amsart}
\usepackage{amsfonts}
\usepackage{amsmath}
\usepackage{amssymb}
\usepackage{amsthm}
\usepackage{float}
\usepackage{mathrsfs}
\usepackage{colonequals}
\usepackage[font=small,labelfont=bf]{caption}
\usepackage[left=1in,right=1in,bottom=1in,top=1in]{geometry}
\usepackage[pdfpagelabels,hyperindex,colorlinks=true,linkcolor=blue,urlcolor=magenta,citecolor=green]{hyperref}
\usepackage{graphicx}
\linespread{1.7}
\emergencystretch=1em

\newcommand \anglePower [2]{\langle #1 \rangle \sp{#2}}
\newcommand \bernoulli [2][B] {{#1}\sb{#2}}
\newcommand \curvePower [2]{\{#1\}\sp{#2}}
\newcommand \coeffA [3][A] {{\mathbf{#1}} \sb{#2,#3}}
\newcommand \polynomialP [4][P]{{\mathbf{#1}}\sp{#2} \sb{#3}(#4)}

% ordinary derivatives
\newcommand \derivative [2] {\frac{d}{d #2} #1}                              % 1 - function; 2 - variable;
\newcommand \pderivative [2] {\frac{\partial #1}{\partial #2}}               % 1 - function; 2 - variable;
\newcommand \qderivative [1] {D_{q} #1}                                      % 1 - function
\newcommand \nqderivative [1] {D_{n,q} #1}                                   % 1 - function
\newcommand \qpowerDerivative [1] {\mathcal{D}_q #1}                         % 1 - function;
\newcommand \finiteDifference [1] {\Delta #1}                                % 1 - function;
\newcommand \pTsDerivative [2] {\frac{\partial #1}{\Delta #2}}               % 1 - function; 2 - variable;

% high order derivatives
\newcommand \derivativeHO [3] {\frac{d^{#3}}{d {#2}^{#3}} #1}                % 1 - function; 2 - variable; 3 - order
\newcommand \pderivativeHO [3]{\frac{\partial^{#3}}{\partial {#2}^{#3}} #1}
\newcommand \qderivativeHO [2] {D_{q}^{#2} #1}                               % 1 - function; 2 - order
\newcommand \qpowerDerivativeHO [2] {\mathcal{D}_{q}^{#2} #1}                % 1 - function; 2 - order
\newcommand \finiteDifferenceHO [2] {\Delta^{#2} #1}                         % 1 - function; 2 - order
\newcommand \pTsDerivativeHO [3] {\frac{\partial^{#3}}{\Delta {#2}^{#3}} #1} % 1 - function; 2 - variable;

\newtheorem{thm}{Theorem}[section]
\newtheorem{cor}[thm]{Corollary}
\newtheorem{lem}[thm]{Lemma}
\newtheorem{examp}[thm]{Example}

\numberwithin{equation}{section}

\title[Polynomial identity involving Binomial Theorem and Faulhaber's formula]
{Polynomial identity involving Binomial Theorem and Faulhaber's formula}
\author[Petro Kolosov]{Petro Kolosov}
\email{kolosovp94@gmail.com}
\keywords{
    Binomial theorem,
    Polynomial identities,
    Binomial coefficients,
    Bernoulli numbers,
    Pascal's triangle,
    Faulhaber's formula
}
\urladdr{https://kolosovpetro.github.io}
\subjclass[2010]{26E70, 05A30}
\date{\today}
\hypersetup{
    pdftitle={Derivation of A Coefficients},
    pdfsubject={
        Binomial theorem,
        Polynomial identities,
        Binomial coefficients,
        Bernoulli numbers,
        Pascal's triangle,
        Faulhaber's formula},
    pdfauthor={Petro Kolosov},
    pdfkeywords={
        Binomial theorem,
        Polynomial identities,
        Binomial coefficients,
        Bernoulli numbers,
        Pascal's triangle,
        Faulhaber's formula}
}
\begin{document}

    \maketitle

    \tableofcontents


    \section{Approach via a system of linear equations}\label{sec:approach-via-system-of-linear-equations}
    One approach to prove the conjecture was proposed by Albert Tkaczyk in his series of articles (two references).
The essence of the approach lays in construction and solving of the particular system of linear equations.
Such system linear equations is constructed using Binomial theorem and Faulhaber's formula
that allows us to find closed froms of power sums as part of identity (reference to equation).
Consider the case of equation (equation reference) for $m=1$
    \begin{examp}
    Let be $m=1$ so that we have the following relation defined by~\eqref{eq:arbitrary-relation}
    \begin{equation*}
        \coeffA{m}{0} n + \coeffA{m}{1} \left[ \frac{1}{6} (-n + n^3) \right] -n^3 = 0
    \end{equation*}
    Multiplying by $6$ right-hand side and left-hand side, we get
    \begin{equation*}
        6\coeffA{1}{0} n + \coeffA{1}{1} (-n + n^3) - 6n^3 = 0
    \end{equation*}
    Opening brackets and rearranging the terms gives
    \begin{equation*}
        6 \coeffA{1}{0} - \coeffA{1}{1} n + \coeffA{1}{1} n^3 - 6n^3 = 0
    \end{equation*}
    Combining the common terms yields
    \begin{equation*}
        n(6\coeffA{1}{0} - \coeffA{1}{1}) + n^3 (\coeffA{1}{1} - 6) = 0
    \end{equation*}
    Therefore, the system of linear equations follows
    \begin{equation*}
        \begin{cases}
            6 \coeffA{1}{0} - \coeffA{1}{1} = 0 \\
            \coeffA{1}{1} - 6 = 0
        \end{cases}
    \end{equation*}
    Solving it, we get
    \begin{equation*}
        \begin{cases}
            \coeffA{1}{1} = 6 \\
            \coeffA{1}{0} = 1
        \end{cases}
    \end{equation*}
    So that odd-power identity~\eqref{eq:odd-power-identity} holds
    \begin{equation*}
        n^3 = \sum_{k=1}^{n} 6k(n-k) + 1
    \end{equation*}
    It is also clearly seen why the above identity is true evaluating the terms $6k(n-k) + 1$ over $0 \leq k \leq n$ as
    the following table shows
    \begin{table}[H]
        \setlength\extrarowheight{-6pt}
        \begin{tabular}{c|cccccccc}
            $n/k$ & 0 & 1  & 2  & 3  & 4  & 5  & 6  & 7 \\
            \hline
            0     & 1 &    &    &    &    &    &    &   \\
            1     & 1 & 1  &    &    &    &    &    &   \\
            2     & 1 & 7  & 1  &    &    &    &    &   \\
            3     & 1 & 13 & 13 & 1  &    &    &    &   \\
            4     & 1 & 19 & 25 & 19 & 1  &    &    &   \\
            5     & 1 & 25 & 37 & 37 & 25 & 1  &    &   \\
            6     & 1 & 31 & 49 & 55 & 49 & 31 & 1  &   \\
            7     & 1 & 37 & 61 & 73 & 73 & 61 & 37 & 1
        \end{tabular}
        \caption{Values of $6k(n-k) + 1$.
        See the OEIS entry: \href{https://oeis.org/A287326}{\texttt{A287326}}~\cite{kolosov2017third}.}
        \label{tab:table-row-sums-gives-cubes}
    \end{table}
\end{examp}

    \begin{examp}
    Let be $m=2$ so that we have the following relation defined by~\eqref{eq:arbitrary-relation}
    \begin{equation*}
        \coeffA{m}{0} n
        + \coeffA{m}{1} \left[ \frac{1}{6} (-n + n^3) \right]
        + \coeffA{m}{2} \left[ \frac{1}{30} (-n + n^5) \right] - n^5 = 0
    \end{equation*}
    Multiplying by $30$ right-hand side and left-hand side, we get
    \begin{equation*}
        30 \coeffA{2}{0} n + 5 \coeffA{2}{1} (-n + n^3) + \coeffA{2}{2} (-n + n^5) - 30n^5 = 0
    \end{equation*}
    Opening brackets and rearranging the terms gives
    \begin{equation*}
        30 \coeffA{2}{0} - 5 \coeffA{2}{1} n + 5 \coeffA{2}{1} n^3 - \coeffA{2}{2} n + \coeffA{2}{2} n^5 - 30n^5 = 0
    \end{equation*}
    Combining the common terms yields
    \begin{equation*}
        n (30 \coeffA{2}{0} - 5 \coeffA{2}{1} - \coeffA{2}{2}) + 5 \coeffA{2}{1} n^3 + n^5 (\coeffA{2}{2} - 30) = 0
    \end{equation*}
    Therefore, the system of linear equations follows
    \begin{equation*}
        \begin{cases}
            30 \coeffA{2}{0} - 5 \coeffA{2}{1} - \coeffA{2}{2} = 0 \\
            \coeffA{2}{1} = 0 \\
            \coeffA{2}{2} - 30 = 0
        \end{cases}
    \end{equation*}
    Solving it, we get
    \begin{equation*}
        \begin{cases}
            \coeffA{2}{2} = 30 \\
            \coeffA{2}{1} = 0 \\
            \coeffA{2}{0} = 1
        \end{cases}
    \end{equation*}
    So that odd-power identity~\eqref{eq:odd-power-identity} holds
    \begin{equation*}
        n^5 = \sum_{k=1}^{n} 30k^2(n-k)^2 + 1
    \end{equation*}
    It is also clearly seen
    why the above identity is true evaluating the terms $30k^2(n-k)^2 + 1$ over $0 \leq k \leq n$ as
    the following table shows
    \begin{table}[H]
        \setlength\extrarowheight{-6pt}
        \begin{tabular}{c|cccccccc}
            $n/k$ & 0 & 1    & 2    & 3    & 4    & 5    & 6    & 7 \\
            \hline
            0     & 1 &      &      &      &      &      &      &   \\
            1     & 1 & 1    &      &      &      &      &      &   \\
            2     & 1 & 31   & 1    &      &      &      &      &   \\
            3     & 1 & 121  & 121  & 1    &      &      &      &   \\
            4     & 1 & 271  & 481  & 271  & 1    &      &      &   \\
            5     & 1 & 481  & 1081 & 1081 & 481  & 1    &      &   \\
            6     & 1 & 751  & 1921 & 2431 & 1921 & 751  & 1    &   \\
            7     & 1 & 1081 & 3001 & 4321 & 4321 & 3001 & 1081 & 1
        \end{tabular}
        \caption{Values of $30k^2(n-k)^2 + 1$.
        See the OEIS entry \href{https://oeis.org/A300656}{\texttt{A300656}}~\cite{kolosov2018fifth}}
        \label{tab:row-sums-gives-fifth-power}
    \end{table}
\end{examp}

    \begin{examp}
    Let be $m=3$ so that we have the following relation defined by~\eqref{eq:arbitrary-relation}
    \begin{equation*}
        \coeffA{m}{0} n
        + \coeffA{m}{1} \left[ \frac{1}{6} (-n + n^3) \right]
        + \coeffA{m}{2} \left[ \frac{1}{30} (-n + n^5) \right]
        + \coeffA{m}{3} \left[ \frac{1}{420} (-10 n + 7 n^3 + 3 n^7) \right] - n^7 = 0
    \end{equation*}
    Multiplying by $420$ right-hand side and left-hand side, we get
    \begin{equation*}
        420 \coeffA{3}{0} n + 70 \coeffA{2}{1} (-n + n^3) + 14 \coeffA{2}{2} (-n + n^5) + \coeffA{3}{3} (-10 n + 7 n^3 + 3 n^7) - 420n^7 = 0
    \end{equation*}
    Opening brackets and rearranging the terms gives
    \begin{equation*}
        \begin{split}
            420 \coeffA{3}{0} n
            &- 70 \coeffA{3}{1} + 70 \coeffA{3}{1} n^3 - 14 \coeffA{3}{2} n + 14 \coeffA{3}{2} n^5 \\
            &- 10 \coeffA{3}{3} n + 7 \coeffA{3}{3} n^3 + 3 \coeffA{3}{3} n^7 - 420n^7 = 0
        \end{split}
    \end{equation*}
    Combining the common terms yields
    \begin{equation*}
        \begin{split}
            &n (420 \coeffA{3}{0} - 70 \coeffA{3}{1} - 14 \coeffA{3}{2} - 10 \coeffA{3}{3}) \\
            &+ n^3 (70 \coeffA{3}{1} + 7 \coeffA{3}{3})
            + n^5 14 \coeffA{3}{2}
            + n^7 (3 \coeffA{3}{3} - 420)
            = 0
        \end{split}
    \end{equation*}
    Therefore, the system of linear equations follows
    \begin{equation*}
        \begin{cases}
            420 \coeffA{3}{0} - 70 \coeffA{3}{1} - 14 \coeffA{3}{2} - 10 \coeffA{3}{3} = 0 \\
            70 \coeffA{3}{1} + 7 \coeffA{3}{3} = 0 \\
            \coeffA{3}{2} - 30 = 0 \\
            3 \coeffA{3}{3} - 420 = 0
        \end{cases}
    \end{equation*}
    Solving it, we get
    \begin{equation*}
        \begin{cases}
            \coeffA{3}{3} = 140 \\
            \coeffA{3}{2} = 0 \\
            \coeffA{3}{1} = -\frac{7}{70} \coeffA{3}{3} = -14 \\
            \coeffA{3}{0} = \frac{(70 \coeffA{3}{1} + 10 \coeffA{3}{3})}{420} = 1
        \end{cases}
    \end{equation*}
    So that odd-power identity~\eqref{eq:odd-power-identity} holds
    \begin{equation*}
        n^7 = \sum_{k=1}^{n} 140 k^3 (n-k)^3 - 14k(n-k) + 1
    \end{equation*}
    It is also clearly seen
    why the above identity is true evaluating the terms $140 k^3 (n-k)^3 - 14k(n-k) + 1$ over $0 \leq k \leq n$ as
    the OEIS sequence \href{https://oeis.org/A300785}{\textit{A300785}}~\cite{oeis_numerical_triangle_row_sums_give_seventh_powers} shows
    \input{figures/04_fig_triangle_row_sums_give_seventh_power}
\end{examp}

    \begin{examp}
    Let be fixed $m=4$ so that we have the following relation defined by (eqref)
    \begin{equation*}
        \begin{split}
            \coeffA{m}{0} n
            &+ \coeffA{m}{1} \left[ \frac{1}{6} (-n + n^3) \right]
            + \coeffA{m}{2} \left[ \frac{1}{30} (-n + n^5) \right]
            + \coeffA{m}{3} \left[ \frac{1}{420} (-10 n + 7 n^3 + 3 n^7) \right] \\
            &+ \coeffA{m}{4} \left[ \frac{1}{630} (-21 n + 20 n^3 + n^9) \right] - n^9 = 0
        \end{split}
    \end{equation*}
    Multiplying by $630$ both, right hand side and left hand side, we get
    \begin{equation*}
        \begin{split}
            630 \coeffA{4}{0} n
            &+ 105 \coeffA{4}{1} (-n + n^3) + 21 \coeffA{4}{2} (-n + n^5) \\
            &+ \frac{3}{2} \coeffA{4}{3} (-10 n + 7 n^3 + 3 n^7) + \coeffA{4}{4} (-21 n + 20 n^3 + n^9) - 630 n^9 = 0
        \end{split}
    \end{equation*}
    Opening brackets and rearranging the terms gives
    \begin{equation*}
        \begin{split}
            630 \coeffA{4}{0} n
            &- 105 \coeffA{4}{1} n + 105 \coeffA{4}{1} n ^3 - 21 \coeffA{4}{2} n + 21 \coeffA{4}{2} n^5 \\
            &- \frac{3}{2} \coeffA{4}{3} \cdot 10n + \frac{3}{2} \coeffA{4}{3} \cdot 7n^3 + \frac{3}{2} \coeffA{4}{3} \cdot 3n^7 \\
            &-21 \coeffA{4}{4} n + 20 \coeffA{4}{4} n^3 + \coeffA{4}{4} n ^9 - 630 n^9 = 0
        \end{split}
    \end{equation*}
    Combining the terms yields
    \begin{equation*}
        \begin{split}
            n (420 \coeffA{3}{0} - 70 \coeffA{3}{1} - 14 \coeffA{3}{2} - 10 \coeffA{3}{3})
            + n^3 (70 \coeffA{3}{1} + 7 \coeffA{3}{3})
            + n^5 14 \coeffA{3}{2}
            + n^7 (3 \coeffA{3}{3} - 420)
            = 0
        \end{split}
    \end{equation*}
    Therefore, the system of linear equations follows
    \begin{equation*}
        \begin{cases}
            420 \coeffA{3}{0} - 70 \coeffA{3}{1} - 14 \coeffA{3}{2} - 10 \coeffA{3}{3} = 0 \\
            70 \coeffA{3}{1} + 7 \coeffA{3}{3} = 0 \\
            \coeffA{3}{2} - 30 = 0 \\
            3 \coeffA{3}{3} - 420 = 0
        \end{cases}
    \end{equation*}
    Solving it we get
    \begin{equation*}
        \begin{cases}
            \coeffA{3}{3} = 140 \\
            \coeffA{3}{2} = 0 \\
            \coeffA{3}{1} = -\frac{7}{70} \coeffA{3}{3} = -14 \\
            \coeffA{3}{0} = \frac{(70 \coeffA{3}{1} + 10 \coeffA{3}{3})}{420} = 1
        \end{cases}
    \end{equation*}
\end{examp}
%
%
%    \section{Coefficients derivation}\label{sec:coefficients-derivation}
%    Consider the Faulhaber's formula
\begin{equation}
    \sum_{k=1}^{n} k^{p} = \frac{1}{p+1}\sum_{j=0}^{p} \binom{p+1}{j} \bernoulli{j} n^{p+1-j}\label{eq:equation3}
\end{equation}
it is very important to note that summation bound is $p$ while binomial coefficient upper bound is $p+1$.
It means that we cannot skip summation bounds unless we do some trick as
\begin{equation}
    \begin{split}
        \sum_{k=1}^{n} k^{p} &= \frac{1}{p+1}\sum_{j=0}^{p} \binom{p+1}{j} \bernoulli{j} n^{p+1-j} \\
        &= \left[ \frac{1}{p+1}\sum_{j=0}^{p+1} \binom{p+1}{j} \bernoulli{j} n^{p+1-j} \right] - \bernoulli{p+1} \\
        &= \left[ \frac{1}{p+1}\sum_{j} \binom{p+1}{j} \bernoulli{j} n^{p+1-j} \right] - \bernoulli{p+1} \\
    \end{split}\label{eq:faulhaber-formula}
\end{equation}
Using Faulhaber's formula
$\sum_{k=1}^{n} k^{p} = \left[ \frac{1}{p+1}\sum_{j} \binom{p+1}{j} \bernoulli{j} n^{p+1-j} \right] - \bernoulli{p+1}$
we get
\begin{equation}
    \begin{split}
        \sum_{k=1}^{n} k^{r} (n-k)^{r}
        &= \sum_{t=0}^{r} (-1)^t \binom{r}{t} n^{r-t} \sum_{k=1}^{n} k^{t+r} \\
        &= \sum_{t=0}^{r} (-1)^t \binom{r}{t} n^{r-t} \left[ \frac{1}{t+r+1} \sum_{j} \binom{t+r+1}{j} \bernoulli{j} n^{t+r+1-j} - \bernoulli{t+r+1} \right] \\
        &= \sum_{t=0}^{r} \binom{r}{t} \left[ \frac{(-1)^t}{t+r+1} \sum_{j} \binom{t+r+1}{j} \bernoulli{j} n^{2r+1-j} - \bernoulli{t+r+1} n^{r-t} \right] \\
        &= \sum_{t=0}^{r} \binom{r}{t} \frac{(-1)^t}{t+r+1} \sum_{j} \binom{t+r+1}{j} \bernoulli{j} n^{2r+1-j} - \sum_{t=0}^{r} \binom{r}{t} \frac{(-1)^t}{t+r+1} \bernoulli{t+r+1} n^{r-t} \\
        &= \sum_{j} \sum_{t} \binom{r}{t} \frac{(-1)^t}{t+r+1} \binom{t+r+1}{j} \bernoulli{j} n^{2r+1-j} - \sum_{t=0}^{r} \binom{r}{t} \frac{(-1)^t}{t+r+1} \bernoulli{t+r+1} n^{r-t} \\
        &= \sum_{j} \bernoulli{j} n^{2r+1-j} \sum_{t} \binom{r}{t} \frac{(-1)^t}{t+r+1} \binom{t+r+1}{j} - \sum_{t=0}^{r} \binom{r}{t} \frac{(-1)^t}{t+r+1} \bernoulli{t+r+1} n^{r-t}
    \end{split}\label{eq:equation4}
\end{equation}
Now, we notice that
\begin{equation}
    \sum_{t} \binom{r}{t} \frac{(-1)^t}{r+t+1} \binom{r+t+1}{j}
    =\begin{cases}
         \frac{1}{(2r+1) \binom{2r}r}, & \text{if } j=0;\\
         \frac{(-1)^r}{j} \binom{r}{2r-j+1}, & \text{if } j>0.
    \end{cases}\label{eq:combinatorial-identity}
\end{equation}
In particular, the last sum is zero for $0< t \leq j$.
So taking $j=0$ we have
\begin{equation}
    \begin{split}
        \sum_{k=1}^{n} k^{r} (n-k)^{r}
        &= \frac{1}{(2r+1) \binom{2r}r} n^{2r+1} + \left[ \sum_{j \geq 1} \bernoulli{j} n^{2r+1-j} \sum_{t} \binom{r}{t} \frac{(-1)^t}{t+r+1} \binom{t+r+1}{j} \right] \\
        &- \left[ \sum_{t=0}^{r} \binom{r}{t} \frac{(-1)^t}{t+r+1} \bernoulli{t+r+1} n^{r-t} \right]
    \end{split}\label{eq:equation5}
\end{equation}
Now let's simplify the double summation
\begin{equation}
    \begin{split}
        \sum_{k=1}^{n} k^{r} (n-k)^{r}
        &= \frac{1}{(2r+1) \binom{2r}r} n^{2r+1}
        + \underbrace{\left[ \sum_{j \geq 1} \frac{(-1)^r}{j} \binom{r}{2r-j+1} \bernoulli{j} n^{2r+1-j} \right]}_{(\star)} \\
        &- \underbrace{\left[ \sum_{t=0}^{r} \binom{r}{t} \frac{(-1)^t}{t+r+1} \bernoulli{t+r+1} n^{r-t} \right]}_{(\diamond)}
    \end{split}\label{eq:equation6}
\end{equation}
Hence, introducing $\ell=2r-j+1$ to $(\star)$ and $\ell=r-t$ to $(\diamond)$, we get
\begin{equation}
    \begin{split}
        \sum_{k=1}^{n} k^{r} (n-k)^{r}
        &= \frac{1}{(2r+1) \binom{2r}r} n^{2r+1}
        + \left[ \sum_{\ell} \frac{(-1)^r}{2r+1-\ell} \binom{r}{\ell} \bernoulli{2r+1-\ell} n^{\ell} \right] \\
        &- \left[ \sum_{\ell} \binom{r}{\ell} \frac{(-1)^{r-\ell}}{2r+1-\ell} \bernoulli{2r+1-\ell} n^{\ell} \right]\\
        &= \frac{1}{(2r+1) \binom{2r}r} n^{2r+1} + 2 \sum_{\mathrm{odd \; \ell}} \frac{(-1)^r}{2r+1-\ell} \binom{r}{\ell} \bernoulli{2r+1-\ell} n^{\ell}
    \end{split}\label{eq:polynomial-sum-1}
\end{equation}
Using the definition of $\coeffA{m}{r}$, we obtain the following identity for polynomials in $n$
\begin{equation}
    \label{eq:equation}
    \sum_{r} \coeffA{m}{r} \frac{1}{(2r+1) \binom{2r}r} n^{2r+1}
    + 2 \sum_{r} \coeffA{m}{r} \sum_{\mathrm{odd \; \ell}} \frac{(-1)^r}{2r+1-\ell} \binom{r}{\ell} \bernoulli{2r+1-\ell} n^{\ell}
    \equiv n^{2m+1}
\end{equation}
Replacing odd $\ell$ by $d$ we get
\begin{equation}
    \begin{split}
        &\sum_{r} \coeffA{m}{r} \frac{1}{(2r+1) \binom{2r}r} n^{2r+1}
        + 2 \sum_{r} \coeffA{m}{r} \sum_{d} \frac{(-1)^r}{2r-2d} \binom{r}{2d+1} \bernoulli{2r-2d} n^{2d+1}
        \equiv n^{2m+1} \\
        &\sum_{r} \coeffA{m}{r} \left[ \frac{1}{(2r+1) \binom{2r}r} n^{2r+1} \right]
        + 2 \sum_{r} \coeffA{m}{r} \left[ \sum_{d} \frac{(-1)^r}{2r-2d} \binom{r}{2d+1} \bernoulli{2r-2d} n^{2d+1} \right]
        - n^{2m+1} = 0 \\
        &\sum_{r=0}^{m} \coeffA{m}{r} \left[ \frac{1}{(2r+1) \binom{2r}r} n^{2r+1} \right]
        + 2 \sum_{r=0}^{m} \coeffA{m}{r} \left[ \sum_{d=0}^{(r-1)/2} \frac{(-1)^r}{2r-2d} \binom{r}{2d+1} \bernoulli{2r-2d} n^{2d+1} \right]
        - n^{2m+1} = 0 \\
    \end{split}\label{eq:equation7}
\end{equation}
Taking the coefficient of $n^{2m+1}$ in~\eqref{eq:equation7}, we get
\begin{equation}
    \coeffA{m}{m} = (2m+1)\binom{2m}{m}\label{eq:equation8}
\end{equation}
and taking the coefficient of $x^{2d+1}$ for an integer $d$ in the range $m/2 \leq d < m$, we get
\begin{equation}
    \coeffA{m}{d} = 0\label{eq:equation9}
\end{equation}
Taking the coefficient of $n^{2d+1}$ for $d$ in the range $m/4 \leq d < m/2$ we get
\begin{equation}
    \coeffA{m}{d} \frac{1}{(2d+1) \binom{2d}{d}}
    +2 \underbrace{(2m+1) \binom{2m}{m}}_{\coeffA{m}{m}} \binom{m}{2d+1} \frac{(-1)^m}{2m-2d} \bernoulli{2m-2d} = 0,\label{eq:equation10}
\end{equation}
i.e
\begin{equation}
    \coeffA{m}{d} = (-1)^{m-1} \frac{(2m+1)!}{d!d!m!(m-2d-1)!} \frac{1}{m-d} \bernoulli{2m-2d}\label{eq:equation11}
\end{equation}
Continue similarly we can express $\coeffA{m}{r}$ for each integer $r$ in range $m/2^{s+1}\leq r < m/2^s$
(iterating consecutively $s=1,2,\ldots$) via previously determined values of $\coeffA{m}{d}$ as follows
\begin{equation}
    \coeffA{m}{r} =
    (2r+1) \binom{2r}{r} \sum_{d \geq 2r+1}^{m} \coeffA{m}{d} \binom{d}{2r+1} \frac{(-1)^{d-1}}{d-r}
    \bernoulli{2d-2r}\label{eq:equation12}
\end{equation}
Finally, the coefficient $\coeffA{m}{r}$ is defined recursively as
\begin{equation}
    \label{eq:def_coeff_a}
    \coeffA{m}{r} \colonequals
    \begin{cases}
    (2r+1)
        \binom{2r}{r}, & \text{if } r=m; \\
        (2r+1) \binom{2r}{r} \sum_{d \geq 2r+1}^{m} \coeffA{m}{d} \binom{d}{2r+1} \frac{(-1)^{d-1}}{d-r}
        \bernoulli{2d-2r}, & \text{if } 0 \leq r<m; \\
        0, & \text{if } r<0 \text{ or } r>m,
    \end{cases}
\end{equation}
where $\bernoulli{t}$ are Bernoulli numbers.
It is assumed that $\bernoulli{1}=\frac{1}{2}$.
\begin{examp}
    Example for $\coeffA{m}{r}$ for $m=2$.
    First we get $\coeffA{2}{2}$
    \begin{equation*}
        \coeffA{m}{m} = 5\binom{4}{2}=30
    \end{equation*}
    Then $\coeffA{2}{1} = 0$ because $\coeffA{m}{d}$ is zero in the range $m/2 \leq d < m$ means that zero for $d$ in
    $1 \leq d < 2$.
    Finally, the $\coeffA{2}{0}$ is
    \begin{equation*}
        \coeffA{2}{0} = 1 \binom{0}{0} \sum_{d \geq 1} \coeffA{2}{d} \binom{d}{1} \frac{(-1)^{d-1}}{d} \bernoulli{2d}
        = \coeffA{2}{2} 2 \frac{-1}{2} \bernoulli{4} = 1.
    \end{equation*}
\end{examp}
\begin{examp}
    Example for $\coeffA{m}{r}$ for $m=3$.
    First we get $\coeffA{3}{3}$
    \begin{equation*}
        \coeffA{m}{m} = 7 \binom{6}{3}= 140
    \end{equation*}
    Then $\coeffA{3}{2} = 0$ because $\coeffA{m}{d}$ is zero in the range $m/2 \leq d < m$ means that zero for $d$ in
    $2 \leq d < 3$.
    The $\coeffA{3}{1}$ coefficient is non-zero and calculated as
    \begin{equation*}
        \begin{split}
            \coeffA{3}{1}
            &= 3 \binom{2}{1} \sum_{d \geq 3} \coeffA{3}{d} \binom{d}{3} \frac{(-1)^{d-1}}{d-1} \bernoulli{2d-2} \\
            &= 3 \binom{2}{1} \coeffA{3}{3} \binom{3}{3} \frac{(-1)^2}{2} \bernoulli{4}
            = 3 \cdot 140 \cdot (-\frac{1}{30}) = -14
        \end{split}
    \end{equation*}
    Finally $\coeffA{3}{0}$ coefficient is
    \begin{equation*}
        \begin{split}
            \coeffA{3}{0}
            &= 1 \binom{0}{0} \sum_{d \geq 1} \coeffA{3}{d} \binom{d}{1} \frac{(-1)^{d-1}}{d} \bernoulli{2d}
            = \sum_{d \geq 1} \coeffA{3}{d} \binom{d}{1} \frac{(-1)^{d-1}}{d} \bernoulli{2d} \\
            & = \coeffA{3}{1} \binom{1}{1} \frac{(-1)^{1-1}}{1} \bernoulli{2}
            + \coeffA{3}{2} \binom{2}{1} \frac{(-1)^{2-1}}{2} \bernoulli{4}
            + \coeffA{3}{3} \binom{3}{1} \frac{(-1)^{3-1}}{3} \bernoulli{6} \\
            & = \coeffA{3}{1} \bernoulli{2} - 2 \coeffA{3}{2} \frac{1}{2} \bernoulli{4}
            + 3 \coeffA{3}{3} \frac{1}{3} \bernoulli{6} \\
            &= \frac{1}{6} \coeffA{3}{1}
            + \coeffA{3}{2} \frac{1}{30}
            + \coeffA{3}{3} \frac{1}{42} \\
            & = \frac{-14}{6} + \frac{140}{42} = 1
        \end{split}
    \end{equation*}
\end{examp}
\begin{examp}
    Example for $\coeffA{m}{r}$ for $m=4$.
    First we get $\coeffA{4}{4}$
    \begin{equation*}
        \coeffA{4}{4} = 9 \binom{8}{4}= 630
    \end{equation*}
    Then $\coeffA{4}{2} = 0, \; \coeffA{4}{3} = 0$
    because $\coeffA{m}{d}$ is zero in the range $m/2 \leq d < m$ means that zero for $d$ in
    $2 \leq d < 4$.
    The $\coeffA{4}{1}$ coefficient is non-zero and calculated as
    \begin{equation*}
        \begin{split}
            \coeffA{4}{1}
            &= 3 \binom{2}{1} \sum_{d \geq 3} \coeffA{4}{d} \binom{d}{3} \frac{(-1)^{d-1}}{d-1} \bernoulli{2d-2} \\
            &= 3 \binom{2}{1} \coeffA{4}{4} \binom{4}{3} \frac{(-1)^3}{3} \bernoulli{6}
            = 3 \cdot 2 \cdot 630 \cdot 4 \cdot (-\frac{1}{3}) \cdot \frac{1}{42} = -120
        \end{split}
    \end{equation*}
    Finally $\coeffA{4}{0}$ coefficient is
    \begin{equation*}
        \begin{split}
            \coeffA{4}{0}
            &= 1 \binom{0}{0} \sum_{d \geq 1} \coeffA{4}{d} \binom{d}{1} \frac{(-1)^{d-1}}{d} \bernoulli{2d}
            = \sum_{d \geq 1} \coeffA{4}{d} \binom{d}{1} \frac{(-1)^{d-1}}{d} \bernoulli{2d} \\
            & = \coeffA{4}{1} \binom{1}{1} \frac{(-1)^{1-1}}{1} \bernoulli{2}
            + \coeffA{4}{2} \binom{2}{1} \frac{(-1)^{2-1}}{2} \bernoulli{4}
            + \coeffA{4}{3} \binom{3}{1} \frac{(-1)^{3-1}}{3} \bernoulli{6}
            + \coeffA{4}{4} \binom{4}{1} \frac{(-1)^{4-1}}{4} \bernoulli{8} \\
%            & = \coeffA{3}{1} \bernoulli{2} - 2 \coeffA{3}{2} \frac{1}{2} \bernoulli{4}
%            + 3 \coeffA{3}{3} \frac{1}{3} \bernoulli{6} \\
            &= \coeffA{4}{1} \frac{1}{6}
            + \coeffA{4}{2} \frac{1}{30}
            + \coeffA{4}{3} \frac{1}{42}
            + \coeffA{4}{4} \frac{1}{30} \\
            & = \frac{-120}{6} + \frac{630}{30} = 1
        \end{split}
    \end{equation*}
\end{examp}
%
%
%    \section{Example 1}\label{sec:example-1}
%    For $m=1$ we have an identity

\begin{equation*}
    \coeffA{m}{0} n + \coeffA{m}{1} \left[ \frac{1}{6} (-n + n^3) \right] - n^3 = 0
\end{equation*}
Multiplying by 6 both parts we get
\begin{equation*}
    6 \coeffA{m}{0} n + \coeffA{m}{1} (-n + n^3)  - 6 n^3 = 0
\end{equation*}

Opening brackets gives
\begin{equation*}
    6 \coeffA{m}{0} n - \coeffA{m}{1} n + \coeffA{m}{1} n^3  - 6 n^3 = 0
\end{equation*}

Arranging terms we get

\begin{equation*}
    n (6\coeffA{m}{0} - \coeffA{m}{1})+ n^3 (\coeffA{m}{1} - 6)  = 0, \quad n \geq 1
\end{equation*}

Hence

\begin{equation*}
    \begin{cases}
        6\coeffA{m}{0} - \coeffA{m}{1} = 0 \\
        \coeffA{m}{1} - 6 = 0
    \end{cases}
\end{equation*}

So that $\coeffA{m}{1}=6 \quad \coeffA{m}{1} = 1$
%
%
%    \section{Example 2}\label{sec:example-2}
%    \input{sections/example2}
%
%
%    \section{Example 3}\label{sec:example-3}
%    \input{sections/example3}
%
%
%    \section{Example 4}\label{sec:example-4}
%    \input{sections/example4}
%
%
%    \section{Example 5}\label{sec:example-5}
%    \input{sections/example5}

%    \section{Examples}\label{sec:examples}
%    Let be an example for $m=2$ of
\begin{equation*}
    \sum_{r=0}^{m} \coeffA{m}{r} \left[ \frac{1}{(2r+1) \binom{2r}r} n^{2r+1} \right]
    + 2 \sum_{r=0}^{m} \coeffA{m}{r} \left[ \sum_{d=0}^{(r-1)/2} \frac{(-1)^r}{2r-2d} \binom{r}{2d+1} \bernoulli{2r-2d} n^{2d+1} \right]
    - n^{2m+1} = 0
\end{equation*}
So that
\begin{equation*}
    \sum_{r=0}^{2} \coeffA{2}{r} \left[ \frac{1}{(2r+1) \binom{2r}r} n^{2r+1} \right]
    + 2 \sum_{r=0}^{2} \coeffA{2}{r} \left[ \sum_{d=0}^{(r-1)/2} \frac{(-1)^r}{2r-2d} \binom{r}{2d+1} \bernoulli{2r-2d} n^{2d+1} \right]
    - n^{5} = 0
\end{equation*}
The sum $\sum_{r=0}^{2} \coeffA{2}{r} \left[ \frac{1}{(2r+1) \binom{2r}r} n^{2r+1} \right]$ in explicit form is
\begin{equation*}
    \sum_{r=0}^{2} \coeffA{2}{r} \left[ \frac{1}{(2r+1) \binom{2r}r} n^{2r+1} \right] =
    \coeffA{2}{0} n
    + \coeffA{2}{1} \left[ \frac{1}{3 \binom{2}{1}} n^{3} \right]
    + \coeffA{2}{2} \left[ \frac{1}{5 \binom{4}{2}} n^{5} \right]
\end{equation*}
Also the sum
$\sum_{r=0}^{2} \coeffA{2}{r} \left[ \sum_{d=0}^{(r-1)/2} \frac{(-1)^r}{2r-2d} \binom{r}{2d+1} \bernoulli{2r-2d} n^{2d+1} \right]$
is
\begin{equation*}
    \begin{split}
        &\sum_{r=0}^{2} \coeffA{2}{r} \left[ \sum_{d=0}^{(r-1)/2} \frac{(-1)^r}{2r-2d} \binom{r}{2d+1} \bernoulli{2r-2d} n^{2d+1} \right] = \\
        & + \coeffA{2}{0} \left[ \sum_{d=0}^{-1} \frac{1}{-2d} \binom{0}{2d+1} \bernoulli{-2d} n^{2d+1} \right] \\
        & - \coeffA{2}{1} \left[ \sum_{d=0}^{0} \frac{1}{2-2d} \binom{1}{2d+1} \bernoulli{2-2d} n^{2d+1} \right] \\
        & + \coeffA{2}{2} \left[ \sum_{d=0}^{0} \frac{1}{4-2d} \binom{2}{2d+1} \bernoulli{4-2d} n^{2d+1} \right] \\
        & = \coeffA{2}{0} \cdot 0
        - \coeffA{2}{1} \left[ \frac{1}{2} \binom{1}{1} \bernoulli{2} n^{1} \right]
        + \coeffA{2}{2} \left[ \frac{1}{4} \binom{2}{1} \bernoulli{4} n^{1} \right] \\
        & = \coeffA{2}{0} \cdot 0
        - \coeffA{2}{1} \left[ \frac{1}{12} n \right]
        - \coeffA{2}{2} \left[ \frac{1}{60} n \right] \\
        & = - \coeffA{2}{1} \frac{1}{12} n
        - \coeffA{2}{2}  \frac{1}{60} n
    \end{split}
\end{equation*}



%    \section{Faulhaber's formulae}\label{sec:faulhaber-formulae}
%    \begin{equation*}
    \begin{split}
        \sum_{k=1}^{n} k^{1} &= 1/2 (-n + n^2) \\
        \sum_{k=1}^{n} k^{2} &= 1/6 (n - 3 n^2 + 2 n^3) \\
        \sum_{k=1}^{n} k^{3} &= 1/4 (n^2 - 2 n^3 + n^4) \\
        \sum_{k=1}^{n} k^{4} &= 1/30 (-n + 10 n^3 - 15 n^4 + 6 n^5) \\
        \sum_{k=1}^{n} k^{5} &= 1/12 (-n^2 + 5 n^4 - 6 n^5 + 2 n^6) \\
        \sum_{k=1}^{n} k^{6} &= 1/42 (n - 7 n^3 + 21 n^5 - 21 n^6 + 6 n^7) \\
        \sum_{k=1}^{n} k^{7} &= 1/24 (2 n^2 - 7 n^4 + 14 n^6 - 12 n^7 + 3 n^8) \\
        \sum_{k=1}^{n} k^{8} &= 1/90 (-3 n + 20 n^3 - 42 n^5 + 60 n^7 - 45 n^8 + 10 n^9) \\
    \end{split}
\end{equation*}
%
%
%    \section{Binomial power sums formulae}\label{sec:binomial-power-sums-formulae}
%    \begin{equation*}
    \begin{split}
        \sum_{t=0}^{1} (-1)^t \binom{1}{t} n^{1-t} \sum_{k=1}^{n} k^{t+1} =& \frac{1}{6} (-n + n^3) \\
        \sum_{t=0}^{2} (-1)^t \binom{2}{t} n^{2-t} \sum_{k=1}^{n} k^{t+2} =& \frac{1}{30} (-n + n^5) \\
        \sum_{t=0}^{3} (-1)^t \binom{3}{t} n^{3-t} \sum_{k=1}^{n} k^{t+3} =& \frac{1}{420} (-10 n + 7 n^3 + 3 n^7) \\
        \sum_{t=0}^{4} (-1)^t \binom{4}{t} n^{4-t} \sum_{k=1}^{n} k^{t+4} =& \frac{1}{630} (-21 n + 20 n^3 + n^9) \\
        \sum_{t=0}^{5} (-1)^t \binom{5}{t} n^{5-t} \sum_{k=1}^{n} k^{t+5} =& \frac{1}{2772} (-210 n + 231 n^3 - 22 n^5 + n^{11}) \\
        \sum_{t=0}^{6} (-1)^t \binom{6}{t} n^{6-t} \sum_{k=1}^{n} k^{t+6} =& \frac{1}{60060} (-15202 n + 18200 n^3 - 3003 n^5 + 5 n^{13}) \\
        \sum_{t=0}^{7} (-1)^t \binom{7}{t} n^{7-t} \sum_{k=1}^{n} k^{t+7} =& \frac{1}{51480} (-60060 n + 76010 n^3 - 16380 n^5 + 429 n^7 + n^{15}) \\
        \sum_{t=0}^{8} (-1)^t \binom{8}{t} n^{8-t} \sum_{k=1}^{n} k^{t+8} =& \frac{1}{218790} (-1551693 n + 2042040 n^3 - 516868 n^5 + 26520 n^7 + n^{17})
    \end{split}
\end{equation*}
%
%
%    \section{Questions}\label{sec:questions}
%    \begin{enumerate}
    \item Any proof or reference to the relation~\eqref{eq:combinatorial-identity}?
    \item What is the motivation to use~\eqref{eq:faulhaber-formula} version of Faulhaber's formula?
    \item Why is there twice odd $\ell$ in~\eqref{eq:polynomial-sum-1}?
\end{enumerate}


%    \section{Conclusions}\label{sec:conclusions}
%    In this manuscript, we have shown that for every $n\geq 1, \; n,m\in\mathbb{N}$
there are coefficients $\mathbf{A}_{m,0}, \mathbf{A}_{m,1}, \ldots, \mathbf{A}_{m,m}$ such that
the polynomial identity holds
\[
    n^{2m+1} = \sum_{k=1}^{n} \mathbf{A}_{m,0} k^0 (n-k)^0 + \mathbf{A}_{m,1}(n-k)^1
    + \cdots + \mathbf{A}_{m,m} k^m (n-k)^m
\]
In particular, the coefficients $\coeffA{m}{r}$ can be evaluated both ways,
by constructing and solving certain system of linear equations or by deriving recurrence relation;
all these approaches are examined providing examples
in the sections~\ref{sec:approach-via-system-of-linear-equations} and~\ref{sec:approach-via-recursion}.
Moreover, to validate the results,
there are supplementary Mathematica programs provided at~\cite{kolosov2023github}.

%
%    \bibliographystyle{unsrt}
%    \bibliography{CoefficientsADerivationReferences}

\end{document}
