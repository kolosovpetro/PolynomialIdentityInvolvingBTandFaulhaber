\begin{examp}
    Let be $m=3$ so that we have the following relation defined by~\eqref{eq:arbitrary-relation}
    \begin{equation*}
        \coeffA{m}{0} n
        + \coeffA{m}{1} \left[ \frac{1}{6} (-n + n^3) \right]
        + \coeffA{m}{2} \left[ \frac{1}{30} (-n + n^5) \right]
        + \coeffA{m}{3} \left[ \frac{1}{420} (-10 n + 7 n^3 + 3 n^7) \right] - n^7 = 0
    \end{equation*}
    Multiplying by $420$ right-hand side and left-hand side, we get
    \begin{equation*}
        420 \coeffA{3}{0} n + 70 \coeffA{2}{1} (-n + n^3) + 14 \coeffA{2}{2} (-n + n^5) + \coeffA{3}{3} (-10 n + 7 n^3 + 3 n^7) - 420n^7 = 0
    \end{equation*}
    Opening brackets and rearranging the terms gives
    \begin{equation*}
        \begin{split}
            420 \coeffA{3}{0} n
            &- 70 \coeffA{3}{1} + 70 \coeffA{3}{1} n^3 - 14 \coeffA{3}{2} n + 14 \coeffA{3}{2} n^5 \\
            &- 10 \coeffA{3}{3} n + 7 \coeffA{3}{3} n^3 + 3 \coeffA{3}{3} n^7 - 420n^7 = 0
        \end{split}
    \end{equation*}
    Combining the common terms yields
    \begin{equation*}
        \begin{split}
            &n (420 \coeffA{3}{0} - 70 \coeffA{3}{1} - 14 \coeffA{3}{2} - 10 \coeffA{3}{3}) \\
            &+ n^3 (70 \coeffA{3}{1} + 7 \coeffA{3}{3})
            + n^5 14 \coeffA{3}{2}
            + n^7 (3 \coeffA{3}{3} - 420)
            = 0
        \end{split}
    \end{equation*}
    Therefore, the system of linear equations follows
    \begin{equation*}
        \begin{cases}
            420 \coeffA{3}{0} - 70 \coeffA{3}{1} - 14 \coeffA{3}{2} - 10 \coeffA{3}{3} = 0 \\
            70 \coeffA{3}{1} + 7 \coeffA{3}{3} = 0 \\
            \coeffA{3}{2} - 30 = 0 \\
            3 \coeffA{3}{3} - 420 = 0
        \end{cases}
    \end{equation*}
    Solving it, we get
    \begin{equation*}
        \begin{cases}
            \coeffA{3}{3} = 140 \\
            \coeffA{3}{2} = 0 \\
            \coeffA{3}{1} = -\frac{7}{70} \coeffA{3}{3} = -14 \\
            \coeffA{3}{0} = \frac{(70 \coeffA{3}{1} + 10 \coeffA{3}{3})}{420} = 1
        \end{cases}
    \end{equation*}
    So that odd-power identity~\eqref{eq:odd-power-identity} holds
    \begin{equation*}
        n^7 = \sum_{k=1}^{n} 140 k^3 (n-k)^3 - 14k(n-k) + 1
    \end{equation*}
    It is also clearly seen
    why the above identity is true evaluating the terms $140 k^3 (n-k)^3 - 14k(n-k) + 1$ over $0 \leq k \leq n$ as
    the OEIS sequence \href{https://oeis.org/A300785}{\textit{A300785}}~\cite{oeis_numerical_triangle_row_sums_give_seventh_powers} shows
    \begin{table}[H]
    \setlength\extrarowheight{-6pt}
    \begin{tabular}{c|cccccccc}
        $n/k$ & 0 & 1     & 2      & 3      & 4      & 5      & 6     & 7 \\
        \hline
        0     & 1 &       &        &        &        &        &       &   \\
        1     & 1 & 1     &        &        &        &        &       &   \\
        2     & 1 & 127   & 1      &        &        &        &       &   \\
        3     & 1 & 1093  & 1093   & 1      &        &        &       &   \\
        4     & 1 & 3739  & 8905   & 3739   & 1      &        &       &   \\
        5     & 1 & 8905  & 30157  & 30157  & 8905   & 1      &       &   \\
        6     & 1 & 17431 & 71569  & 101935 & 71569  & 17431  & 1     &   \\
        7     & 1 & 30157 & 139861 & 241753 & 241753 & 139861 & 30157 & 1
    \end{tabular}
    \caption{Values of $140 k^3 (n-k)^3 - 14k(n-k) + 1$.
    See the OEIS entry \href{https://oeis.org/A300785}{\texttt{A300785}}~\cite{oeis_numerical_triangle_row_sums_give_seventh_powers}.}
    \label{tab:row-sums-gives-seventh-power}
\end{table}

\end{examp}
