\begin{examp}
    Let be $m=1$ so that we have the following relation defined by (eqref)
    \begin{equation*}
        \coeffA{m}{0} n + \coeffA{m}{1} \left[ \frac{1}{6} (-n + n^3) \right] -n^3 = 0
    \end{equation*}
    Multiplying by $6$ right-hand side and left-hand side, we get
    \begin{equation*}
        6\coeffA{1}{0} n + \coeffA{1}{1} (-n + n^3) - 6n^3 = 0
    \end{equation*}
    Opening brackets and rearranging the terms gives
    \begin{equation*}
        6 \coeffA{1}{0} - \coeffA{1}{1} n + \coeffA{1}{1} n^3 - 6n^3 = 0
    \end{equation*}
    Combining the common terms yields
    \begin{equation*}
        n(6\coeffA{1}{0} - \coeffA{1}{1}) + n^3 (\coeffA{1}{1} - 6) = 0
    \end{equation*}
    Therefore, the system of linear equations follows
    \begin{equation*}
        \begin{cases}
            6 \coeffA{1}{0} - \coeffA{1}{1} = 0 \\
            \coeffA{1}{1} - 6 = 0
        \end{cases}
    \end{equation*}
    Solving it, we get
    \begin{equation*}
        \begin{cases}
            \coeffA{1}{1} = 6 \\
            \coeffA{1}{0} = 1
        \end{cases}
    \end{equation*}
    So that odd-power identity~\eqref{eq:odd-power-identity} holds
    \begin{equation*}
        n^3 = \sum_{k=1}^{n} 6k(n-k) + 1
    \end{equation*}
    It is also clearly seen why the above identity is true evaluating the terms $6k(n-k) + 1$ over $0 \leq k \leq n$ as
    it is shown at~\cite{kolosov2017third}.
\end{examp}