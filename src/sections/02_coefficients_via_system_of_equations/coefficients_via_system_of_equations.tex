One approach to proving the conjecture was proposed by Albert Tkaczyk
in his series of the preprints~\cite{tkaczyk2018problem, tkaczyk2018continuation}
and extended further at~\cite{unusual_identity_for_odd_powers}.
The main idea is to construct and solve a system of linear equations.
Such a system of linear equations is constructed
by expanding the definition of the coefficients $\coeffA{m}{r}$
applying Binomial theorem~\cite{abramowitz1988handbook} and Faulhaber's formula~\cite{beardon1996sums}.
Consider the definition of the coefficients $\coeffA{m}{r}$
\begin{equation}
    n^{2m+1} = \sum_{r=0}^{m} \coeffA{m}{r} \sum_{k=1}^{n} k^{r} (n-k)^{r}\label{eq:odd-power-identity}
\end{equation}
Expanding the $(n-k)^r$ part via Binomial theorem, we get
\begin{equation*}
    \begin{split}
        n^{2m+1} &= \sum_{r=0}^{m} \coeffA{m}{r} \sum_{k=1}^{n} k^{r} (n-k)^{r} \\
        &= \sum_{r=0}^{m} \coeffA{m}{r} \sum_{k=1}^{n} k^{r} \left[ \sum_{t=0}^{r} (-1)^t \binom{r}{t} n^{r-t} k^{t} \right] \\
        &= \sum_{r=0}^{m} \coeffA{m}{r} \left[ \sum_{t=0}^{r} (-1)^t \binom{r}{t} n^{r-t} \sum_{k=1}^{n} k^{t+r} \right] \\
    \end{split}
\end{equation*}
Applying the Faulhaber's formula to the sum $\sum_{k=1}^{n} k^{t+r}$ we get
\begin{equation}
    \begin{split}
        n^{2m+1}
        &= \sum_{r=0}^{m} \coeffA{m}{r} \left[ \sum_{t=0}^{r} (-1)^t \binom{r}{t} n^{r-t} \sum_{k=1}^{n} k^{t+r} \right] \\
        &= \coeffA{m}{0} n  + \coeffA{m}{1} \left[ \frac{1}{6} (-n + n^3) \right] + \coeffA{m}{2} \left[ \frac{1}{30} (-n + n^5) \right] \\
        &+ \coeffA{m}{3} \left[ \frac{1}{420} (-10 n + 7 n^3 + 3 n^7) \right] + \coeffA{m}{4} \left[ \frac{1}{630} (-21 n + 20 n^3 + n^9) \right] \\
        &+ \coeffA{m}{5} \left[ \frac{1}{2772} (-210 n + 231 n^3 - 22 n^5 + n^{11}) \right] \\
        &+ \coeffA{m}{6} \left[ \frac{1}{60060} (-15202 n + 18200 n^3 - 3003 n^5 + 5 n^{13}) \right] \\
        &+ \coeffA{m}{7} \left[ \frac{1}{51480} (-60060 n + 76010 n^3 - 16380 n^5 + 429 n^7 + n^{15}) \right] \\
        &+ \coeffA{m}{8} \left[ \frac{1}{218790} (-1551693 n + 2042040 n^3 - 516868 n^5 + 26520 n^7 + n^{17}) \right] + \cdots
    \end{split}\label{eq:arbitrary-relation}
\end{equation}
Given a fixed integer $m$, the coefficients $\coeffA{m}{r}$ can be determined via a system of linear equations.
Consider an example
