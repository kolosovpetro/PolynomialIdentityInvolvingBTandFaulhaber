\begin{examp}
    Example for $\coeffA{m}{r}$ for $m=4$.
    First we get $\coeffA{4}{4}$
    \begin{equation*}
        \coeffA{4}{4} = 9 \binom{8}{4}= 630
    \end{equation*}
    Then $\coeffA{4}{2} = 0, \; \coeffA{4}{3} = 0$
    because $\coeffA{m}{d}$ is zero in the range $m/2 \leq d < m$ means that zero for $d$ in
    $2 \leq d < 4$.
    The $\coeffA{4}{1}$ coefficient is non-zero and calculated as
    \begin{equation*}
        \begin{split}
            \coeffA{4}{1}
            &= 3 \binom{2}{1} \sum_{d \geq 3} \coeffA{4}{d} \binom{d}{3} \frac{(-1)^{d-1}}{d-1} \bernoulli{2d-2} \\
            &= 3 \binom{2}{1} \coeffA{4}{4} \binom{4}{3} \frac{(-1)^3}{3} \bernoulli{6}
            = 3 \cdot 2 \cdot 630 \cdot 4 \cdot (-\frac{1}{3}) \cdot \frac{1}{42} = -120
        \end{split}
    \end{equation*}
    Finally $\coeffA{4}{0}$ coefficient is
    \begin{equation*}
        \begin{split}
            \coeffA{4}{0}
            &= 1 \binom{0}{0} \sum_{d \geq 1} \coeffA{4}{d} \binom{d}{1} \frac{(-1)^{d-1}}{d} \bernoulli{2d}
            = \sum_{d \geq 1} \coeffA{4}{d} \binom{d}{1} \frac{(-1)^{d-1}}{d} \bernoulli{2d} \\
            & = \coeffA{4}{1} \binom{1}{1} \frac{(-1)^{1-1}}{1} \bernoulli{2}
            + \coeffA{4}{2} \binom{2}{1} \frac{(-1)^{2-1}}{2} \bernoulli{4}
            + \coeffA{4}{3} \binom{3}{1} \frac{(-1)^{3-1}}{3} \bernoulli{6}
            + \coeffA{4}{4} \binom{4}{1} \frac{(-1)^{4-1}}{4} \bernoulli{8} \\
%            & = \coeffA{3}{1} \bernoulli{2} - 2 \coeffA{3}{2} \frac{1}{2} \bernoulli{4}
%            + 3 \coeffA{3}{3} \frac{1}{3} \bernoulli{6} \\
            &= \coeffA{4}{1} \frac{1}{6}
            + \coeffA{4}{2} \frac{1}{30}
            + \coeffA{4}{3} \frac{1}{42}
            + \coeffA{4}{4} \frac{1}{30} \\
            & = \frac{-120}{6} + \frac{630}{30} = 1
        \end{split}
    \end{equation*}
\end{examp}