\begin{example}
    Let be $m=4$ so that first we get $\coeffA{4}{4}$
    \begin{equation*}
        \coeffA{4}{4} = 9 \binom{8}{4}= 630
    \end{equation*}
    Then $\coeffA{4}{3} = 0$ and $\coeffA{4}{2} = 0$
    because $\coeffA{m}{d}$ is zero in the range $m/2 \leq d < m$ means that zero for $d$ in $2 \leq d < 4$.
    The value of the coefficient $\coeffA{4}{1}$ is non-zero and calculated as
    \begin{equation*}
        \begin{split}
            \coeffA{4}{1}
            = \sum_{d \geq 3}^{4} \coeffA{4}{d} \cdot T(d,1)
            = \coeffA{4}{3} \cdot T(3,1) + \coeffA{4}{4} \cdot T(4,1)
            = 630 \cdot \left( -\frac{4}{21} \right)
            = -120
        \end{split}
    \end{equation*}
    Finally, the coefficient $\coeffA{4}{0}$ is
    \begin{equation*}
        \begin{split}
            \coeffA{4}{0}
            = \sum_{d \geq 1}^{4} \coeffA{4}{d} \cdot T(d, 0)
            = \coeffA{4}{1} \cdot T(1, 0) + \coeffA{4}{4} \cdot T(4, 0)
            = -120 \cdot \frac{1}{6} + 630 \cdot \frac{1}{30} = 1
        \end{split}
    \end{equation*}
\end{example}
