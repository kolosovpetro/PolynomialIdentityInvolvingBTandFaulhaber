\documentclass[12pt,letterpaper,oneside,reqno]{amsart}
\usepackage{amsfonts}
\usepackage{amsmath}
\usepackage{amssymb}
\usepackage{amsthm}
\usepackage{float}
\usepackage{mathrsfs}
\usepackage{colonequals}
\usepackage[font=small,labelfont=bf]{caption}
\usepackage[left=1in,right=1in,bottom=1in,top=1in]{geometry}
\usepackage[pdfpagelabels,hyperindex,colorlinks=true,linkcolor=blue,urlcolor=magenta,citecolor=green]{hyperref}
\usepackage{graphicx}
\usepackage{array}
\linespread{1.7}
\emergencystretch=1em
\usepackage{etoolbox}
\apptocmd{\sloppy}{\hbadness 10000\relax}{}{}
\raggedbottom

\newcommand \anglePower [2]{\langle #1 \rangle \sp{#2}}
\newcommand \bernoulli [2][B] {{#1}\sb{#2}}
\newcommand \curvePower [2]{\{#1\}\sp{#2}}
\newcommand \coeffA [3][A] {{\mathbf{#1}} \sb{#2,#3}}
\newcommand \polynomialP [4][P]{{\mathbf{#1}}\sp{#2} \sb{#3}(#4)}

% ordinary derivatives
\newcommand \derivative [2] {\frac{d}{d #2} #1}                              % 1 - function; 2 - variable;
\newcommand \pderivative [2] {\frac{\partial #1}{\partial #2}}               % 1 - function; 2 - variable;
\newcommand \qderivative [1] {D_{q} #1}                                      % 1 - function
\newcommand \nqderivative [1] {D_{n,q} #1}                                   % 1 - function
\newcommand \qpowerDerivative [1] {\mathcal{D}_q #1}                         % 1 - function;
\newcommand \finiteDifference [1] {\Delta #1}                                % 1 - function;
\newcommand \pTsDerivative [2] {\frac{\partial #1}{\Delta #2}}               % 1 - function; 2 - variable;

% high order derivatives
\newcommand \derivativeHO [3] {\frac{d^{#3}}{d {#2}^{#3}} #1}                % 1 - function; 2 - variable; 3 - order
\newcommand \pderivativeHO [3]{\frac{\partial^{#3}}{\partial {#2}^{#3}} #1}
\newcommand \qderivativeHO [2] {D_{q}^{#2} #1}                               % 1 - function; 2 - order
\newcommand \qpowerDerivativeHO [2] {\mathcal{D}_{q}^{#2} #1}                % 1 - function; 2 - order
\newcommand \finiteDifferenceHO [2] {\Delta^{#2} #1}                         % 1 - function; 2 - order
\newcommand \pTsDerivativeHO [3] {\frac{\partial^{#3}}{\Delta {#2}^{#3}} #1} % 1 - function; 2 - variable;

% free foot note
\let\svthefootnote\thefootnote
\newcommand\freefootnote[1]{%
    \let\thefootnote\relax%
    \footnotetext{#1}%
    \let\thefootnote\svthefootnote%
}

\newtheorem{theorem}{Theorem}[section]
\newtheorem{example}[theorem]{Example}
\newtheorem{conjecture}[theorem]{Conjecture}
\newtheorem{definition}[theorem]{Definition}

\numberwithin{equation}{section}

\title[Polynomial identity involving Binomial Theorem and Faulhaber's formula]
{Polynomial identity involving Binomial Theorem and Faulhaber's formula}
\author[Petro Kolosov]{Petro Kolosov}
\email{kolosovp94@gmail.com}
\keywords{
    Binomial theorem,
    Polynomial identities,
    Binomial coefficients,
    Bernoulli numbers,
    Pascal's triangle,
    Faulhaber's formula,
    Polynomials
}
\urladdr{https://kolosovpetro.github.io}
\subjclass[2010]{26E70, 05A30}
\date{\today}
\hypersetup{
    pdftitle={Polynomial identity involving Binomial Theorem and Faulhaber's formula},
    pdfsubject={
        Binomial theorem,
        Polynomial identities,
        Binomial coefficients,
        Bernoulli numbers,
        Pascal's triangle,
        Faulhaber's formula,
        Power function,
        Polynomials,
        Discrete mathematics,
        Combinatorics
    },
    pdfauthor={Petro Kolosov},
    pdfkeywords={
        Binomial theorem,
        Polynomial identities,
        Binomial coefficients,
        Bernoulli numbers,
        Pascal's triangle,
        Faulhaber's formula,
        Power function,
        Polynomials,
        Discrete mathematics,
        Combinatorics
    }
}
\begin{document}
    \begin{abstract}
        Derivation of $\coeffA{m}{r}$ in a simple and explicit manner.
    \end{abstract}

    \maketitle

    \tableofcontents

    \freefootnote{Sources: \url{https://github.com/kolosovpetro/PolynomialIdentityInvolvingBTandFaulhaber}}


    \section{Introduction}\label{sec:introduction}
    Considering the table of forward finite differences of the polynomial $n^3$
\begin{table}[H]
    \begin{center}
        \setlength\extrarowheight{-6pt}
        \begin{tabular}{c|cccc}
            $n$ & $n^3$ & $\Delta(n^3)$ & $\Delta^2(n^3)$ & $\Delta^3(n^3)$ \\
            \hline
            0   & 0     & 1             & 6               & 6               \\
            1   & 1     & 7             & 12              & 6               \\
            2   & 8     & 19            & 18              & 6               \\
            3   & 27    & 37            & 24              & 6               \\
            4   & 64    & 61            & 30              & 6               \\
            5   & 125   & 91            & 36              &                 \\
            6   & 216   & 127           &                 &                 \\
            7   & 343   &               &                 &
        \end{tabular}
    \end{center}
    \caption{Table of finite differences of the polynomial $n^3$.} \label{tab:table}
\end{table}
We can easily observe that finite differences
\footnote{One may assume that it is possible to reach the form $n^{2m+1} = \sum_{k=1}^{n} \mathbf{A}_{m,0} k^0 (n-k)^0 + \mathbf{A}_{m,1}(n-k)^1
+ \cdots + \mathbf{A}_{m,m} k^m (n-k)^m$ simply taking finite differences of the polynomial $n^{2m+1}$ up to order of $2m+1$ 
and interpolating it backwards similarly as shown in~\eqref{eq:cubes_interpolation}.
However, my observations do not provide any evidence of such assumption.
Interestingly enough is that we could have been arrived to the pure differential approach of the relation~\eqref{eq:odd_power_conjecture} then.}
of the polynomial $n^3$ may be expressed according
to the following relation, via rearrangement of the terms
\begin{align}
    \label{eq:cubes_interpolation}
    \begin{split}
        \Delta(0^3) &= 1+6 \cdot 0 \\
        \Delta(1^3) &= 1+6\cdot0+6\cdot1 \\
        \Delta(2^3) &= 1+6\cdot0+6\cdot1+6\cdot2 \\
        \Delta(3^3) &= 1+6\cdot0+6\cdot1+6\cdot2+6\cdot3 \\
        &\; \; \vdots \\
        \Delta(n^3) &= 1+6\cdot0+6\cdot1+6\cdot2+6\cdot3+\cdots+6\cdot n
    \end{split}
\end{align}
Furthermore, the polynomial $n^3$ is equivalent to
\begin{align*}
    n^3 &= [1+6\cdot0]+[1+6\cdot0+6\cdot1]+[1+6\cdot0+6\cdot1+6\cdot2]+\cdots \\
    &+[1+6\cdot0+6\cdot1+6\cdot2+\cdots+6\cdot(n-1)]
\end{align*}
Rearranging the above equation, we get
\[
    n^3 = n +(n-0) \cdot6 \cdot0 + (n-1)\cdot6\cdot1 + (n-2)\cdot6\cdot2 + \cdots+1\cdot6\cdot(n-1)
\]
Therefore, we can consider the polynomial $n^3$ as
\begin{equation}
    \label{eq:cube_identity}
    n^3 = \sum_{k=1}^{n} 6k(n-k) + 1
\end{equation}
Assume that equation~\eqref{eq:cube_identity} has the following implicit form
\begin{equation}
    \label{eq:pattern}
    n^3 = \sum_{k=1}^{n} \coeffA{1}{1} k^1(n-k)^1 + \coeffA{1}{0} k^0(n-k)^0,
\end{equation}
where $\coeffA{1}{1} = 6$ and $\coeffA{1}{0} = 1$, respectively.
Note that here the power of $3$ is actually defined by $2m+1$ where $m=1$.
So, is there a generalization of the relation~\eqref{eq:pattern} for all positive odd powers $2m+1, \; m=0,1,2,\dots$?
Therefore, let us propose a conjecture
\begin{conj}
    For every $n\geq 1, \; n,m\in\mathbb{N}$ there are coefficients $\coeffA{m}{0}, \coeffA{m}{1}, \ldots, \coeffA{m}{m}$ such that
    \begin{equation}
        \label{eq:odd_power_conjecture}
        n^{2m+1} = \sum_{k=1}^{n} \coeffA{m}{0} k^0 (n-k)^0 + \coeffA{m}{1} (n-k)^1
        + \cdots + \coeffA{m}{m} k^m (n-k)^m
    \end{equation}
\end{conj}



    \section{Approach via a system of linear equations}\label{sec:approach-via-system-of-linear-equations}
    One approach to prove the conjecture was proposed by Albert Tkaczyk in his series of preprints (two references).
The essence of the approach lays in construction and solving of the particular system of linear equations.
Such system linear equations is constructed using Binomial theorem and Faulhaber's formula
that allows us to find closed froms of power sums as part of identity (reference to equation).
Consider the polynomial relation
\begin{equation}
    n^{2m+1} = \sum_{r=0}^{m} \coeffA{m}{r} \sum_{k=1}^{n} k^{r} (n-k)^{r}\label{eq:odd-power-identity}
\end{equation}
Expanding the $(n-k)^r$ part via Binomial theorem we get
\begin{equation}
    \begin{split}
        n^{2m+1} &= \sum_{r=0}^{m} \coeffA{m}{r} \sum_{k=1}^{n} k^{r} (n-k)^{r} \\
        &= \sum_{r=0}^{m} \coeffA{m}{r} \sum_{k=1}^{n} k^{r} \left[ \sum_{t=0}^{r} (-1)^t \binom{r}{t} n^{r-t} k^{t} \right] \\
        &= \sum_{r=0}^{m} \coeffA{m}{r} \left[ \sum_{t=0}^{r} (-1)^t \binom{r}{t} n^{r-t} \sum_{k=1}^{n} k^{t+r} \right] \\
    \end{split}\label{eq:equation2}
\end{equation}
For arbitrary $m$ we have
\begin{equation}
    \begin{split}
        n^{2m+1}
        &= \sum_{r=0}^{m} \coeffA{m}{r} \left[ \sum_{t=0}^{r} (-1)^t \binom{r}{t} n^{r-t} \sum_{k=1}^{n} k^{t+r} \right] \\
        &= \coeffA{m}{0} n
        + \coeffA{m}{1} \left[ \frac{1}{6} (-n + n^3) \right]
        + \coeffA{m}{2} \left[ \frac{1}{30} (-n + n^5) \right]
        + \coeffA{m}{3} \left[ \frac{1}{420} (-10 n + 7 n^3 + 3 n^7) \right] \\
        &+ \coeffA{m}{4} \left[ \frac{1}{630} (-21 n + 20 n^3 + n^9) \right]
        + \coeffA{m}{5} \left[ \frac{1}{2772} (-210 n + 231 n^3 - 22 n^5 + n^{11}) \right] \\
        &+ \coeffA{m}{6} \left[ \frac{1}{60060} (-15202 n + 18200 n^3 - 3003 n^5 + 5 n^{13}) \right] \\
        &+ \coeffA{m}{7} \left[ \frac{1}{51480} (-60060 n + 76010 n^3 - 16380 n^5 + 429 n^7 + n^{15}) \right] \\
        &+ \coeffA{m}{8} \left[ \frac{1}{218790} (-1551693 n + 2042040 n^3 - 516868 n^5 + 26520 n^7 + n^{17}) \right] + \cdots
    \end{split}\label{eq:equation17}
\end{equation}
\begin{examp}
    Let be $m=1$ so that we have the following relation defined by (eqref)
    \begin{equation*}
        \coeffA{m}{0} n + \coeffA{m}{1} \left[ \frac{1}{6} (-n + n^3) \right] -n^3 = 0
    \end{equation*}
    Multiplying by $6$ right-hand side and left-hand side, we get
    \begin{equation*}
        6\coeffA{1}{0} n + \coeffA{1}{1} (-n + n^3) - 6n^3 = 0
    \end{equation*}
    Opening brackets and rearranging the terms gives
    \begin{equation*}
        6 \coeffA{1}{0} - \coeffA{1}{1} n + \coeffA{1}{1} n^3 - 6n^3 = 0
    \end{equation*}
    Combining the common terms yields
    \begin{equation*}
        n(6\coeffA{1}{0} - \coeffA{1}{1}) + n^3 (\coeffA{1}{1} - 6) = 0
    \end{equation*}
    Therefore, the system of linear equations follows
    \begin{equation*}
        \begin{cases}
            6 \coeffA{1}{0} - \coeffA{1}{1} = 0 \\
            \coeffA{1}{1} - 6 = 0
        \end{cases}
    \end{equation*}
    Solving it, we get
    \begin{equation*}
        \begin{cases}
            \coeffA{1}{1} = 6 \\
            \coeffA{1}{0} = 1
        \end{cases}
    \end{equation*}
\end{examp}
\begin{examp}
    Let be $m=2$ so that we have the following relation defined by~\eqref{eq:arbitrary-relation}
    \begin{equation*}
        \coeffA{m}{0} n
        + \coeffA{m}{1} \left[ \frac{1}{6} (-n + n^3) \right]
        + \coeffA{m}{2} \left[ \frac{1}{30} (-n + n^5) \right] - n^5 = 0
    \end{equation*}
    Multiplying by $30$ right-hand side and left-hand side, we get
    \begin{equation*}
        30 \coeffA{2}{0} n + 5 \coeffA{2}{1} (-n + n^3) + \coeffA{2}{2} (-n + n^5) - 30n^5 = 0
    \end{equation*}
    Opening brackets and rearranging the terms gives
    \begin{equation*}
        30 \coeffA{2}{0} - 5 \coeffA{2}{1} n + 5 \coeffA{2}{1} n^3 - \coeffA{2}{2} n + \coeffA{2}{2} n^5 - 30n^5 = 0
    \end{equation*}
    Combining the common terms yields
    \begin{equation*}
        n (30 \coeffA{2}{0} - 5 \coeffA{2}{1} - \coeffA{2}{2}) + 5 \coeffA{2}{1} n^3 + n^5 (\coeffA{2}{2} - 30) = 0
    \end{equation*}
    Therefore, the system of linear equations follows
    \begin{equation*}
        \begin{cases}
            30 \coeffA{2}{0} - 5 \coeffA{2}{1} - \coeffA{2}{2} = 0 \\
            \coeffA{2}{1} = 0 \\
            \coeffA{2}{2} - 30 = 0
        \end{cases}
    \end{equation*}
    Solving it, we get
    \begin{equation*}
        \begin{cases}
            \coeffA{2}{2} = 30 \\
            \coeffA{2}{1} = 0 \\
            \coeffA{2}{0} = 1
        \end{cases}
    \end{equation*}
    So that odd-power identity~\eqref{eq:odd-power-identity} holds
    \begin{equation*}
        n^5 = \sum_{k=1}^{n} 30k^2(n-k)^2 + 1
    \end{equation*}
    It is also clearly seen
    why the above identity is true evaluating the terms $30k^2(n-k)^2 + 1$ over $0 \leq k \leq n$ as
    the following table shows
    \begin{table}[H]
    \setlength\extrarowheight{-6pt}
    \begin{tabular}{c|cccccccc}
        $n/k$ & 0 & 1    & 2    & 3    & 4    & 5    & 6    & 7 \\
        \hline
        0     & 1 &      &      &      &      &      &      &   \\
        1     & 1 & 1    &      &      &      &      &      &   \\
        2     & 1 & 31   & 1    &      &      &      &      &   \\
        3     & 1 & 121  & 121  & 1    &      &      &      &   \\
        4     & 1 & 271  & 481  & 271  & 1    &      &      &   \\
        5     & 1 & 481  & 1081 & 1081 & 481  & 1    &      &   \\
        6     & 1 & 751  & 1921 & 2431 & 1921 & 751  & 1    &   \\
        7     & 1 & 1081 & 3001 & 4321 & 4321 & 3001 & 1081 & 1
    \end{tabular}
    \caption{Values of $30k^2(n-k)^2 + 1$.
    See the OEIS entry \href{https://oeis.org/A300656}{\texttt{A300656}}~\cite{oeis_numerical_triangle_row_sums_give_fifth_powers}.}
    \label{tab:row-sums-gives-fifth-power}
\end{table}

\end{examp}

    \begin{examp}
    Let be $m=1$ so that we have the following relation defined by (eqref)
    \begin{equation*}
        \coeffA{m}{0} n + \coeffA{m}{1} \left[ \frac{1}{6} (-n + n^3) \right] -n^3 = 0
    \end{equation*}
    Multiplying by $6$ right-hand side and left-hand side, we get
    \begin{equation*}
        6\coeffA{1}{0} n + \coeffA{1}{1} (-n + n^3) - 6n^3 = 0
    \end{equation*}
    Opening brackets and rearranging the terms gives
    \begin{equation*}
        6 \coeffA{1}{0} - \coeffA{1}{1} n + \coeffA{1}{1} n^3 - 6n^3 = 0
    \end{equation*}
    Combining the common terms yields
    \begin{equation*}
        n(6\coeffA{1}{0} - \coeffA{1}{1}) + n^3 (\coeffA{1}{1} - 6) = 0
    \end{equation*}
    Therefore, the system of linear equations follows
    \begin{equation*}
        \begin{cases}
            6 \coeffA{1}{0} - \coeffA{1}{1} = 0 \\
            \coeffA{1}{1} - 6 = 0
        \end{cases}
    \end{equation*}
    Solving it, we get
    \begin{equation*}
        \begin{cases}
            \coeffA{1}{1} = 6 \\
            \coeffA{1}{0} = 1
        \end{cases}
    \end{equation*}
\end{examp}
    \begin{examp}
    Let be $m=2$ so that we have the following relation defined by~\eqref{eq:arbitrary-relation}
    \begin{equation*}
        \coeffA{m}{0} n
        + \coeffA{m}{1} \left[ \frac{1}{6} (-n + n^3) \right]
        + \coeffA{m}{2} \left[ \frac{1}{30} (-n + n^5) \right] - n^5 = 0
    \end{equation*}
    Multiplying by $30$ right-hand side and left-hand side, we get
    \begin{equation*}
        30 \coeffA{2}{0} n + 5 \coeffA{2}{1} (-n + n^3) + \coeffA{2}{2} (-n + n^5) - 30n^5 = 0
    \end{equation*}
    Opening brackets and rearranging the terms gives
    \begin{equation*}
        30 \coeffA{2}{0} - 5 \coeffA{2}{1} n + 5 \coeffA{2}{1} n^3 - \coeffA{2}{2} n + \coeffA{2}{2} n^5 - 30n^5 = 0
    \end{equation*}
    Combining the common terms yields
    \begin{equation*}
        n (30 \coeffA{2}{0} - 5 \coeffA{2}{1} - \coeffA{2}{2}) + 5 \coeffA{2}{1} n^3 + n^5 (\coeffA{2}{2} - 30) = 0
    \end{equation*}
    Therefore, the system of linear equations follows
    \begin{equation*}
        \begin{cases}
            30 \coeffA{2}{0} - 5 \coeffA{2}{1} - \coeffA{2}{2} = 0 \\
            \coeffA{2}{1} = 0 \\
            \coeffA{2}{2} - 30 = 0
        \end{cases}
    \end{equation*}
    Solving it, we get
    \begin{equation*}
        \begin{cases}
            \coeffA{2}{2} = 30 \\
            \coeffA{2}{1} = 0 \\
            \coeffA{2}{0} = 1
        \end{cases}
    \end{equation*}
    So that odd-power identity~\eqref{eq:odd-power-identity} holds
    \begin{equation*}
        n^5 = \sum_{k=1}^{n} 30k^2(n-k)^2 + 1
    \end{equation*}
    It is also clearly seen
    why the above identity is true evaluating the terms $30k^2(n-k)^2 + 1$ over $0 \leq k \leq n$ as
    the following table shows
    \begin{table}[H]
    \setlength\extrarowheight{-6pt}
    \begin{tabular}{c|cccccccc}
        $n/k$ & 0 & 1    & 2    & 3    & 4    & 5    & 6    & 7 \\
        \hline
        0     & 1 &      &      &      &      &      &      &   \\
        1     & 1 & 1    &      &      &      &      &      &   \\
        2     & 1 & 31   & 1    &      &      &      &      &   \\
        3     & 1 & 121  & 121  & 1    &      &      &      &   \\
        4     & 1 & 271  & 481  & 271  & 1    &      &      &   \\
        5     & 1 & 481  & 1081 & 1081 & 481  & 1    &      &   \\
        6     & 1 & 751  & 1921 & 2431 & 1921 & 751  & 1    &   \\
        7     & 1 & 1081 & 3001 & 4321 & 4321 & 3001 & 1081 & 1
    \end{tabular}
    \caption{Values of $30k^2(n-k)^2 + 1$.
    See the OEIS entry \href{https://oeis.org/A300656}{\texttt{A300656}}~\cite{oeis_numerical_triangle_row_sums_give_fifth_powers}.}
    \label{tab:row-sums-gives-fifth-power}
\end{table}

\end{examp}

    \begin{examp}
    Let be $m=3$ so that we have the following relation defined by~\eqref{eq:arbitrary-relation}
    \begin{equation*}
        \coeffA{m}{0} n
        + \coeffA{m}{1} \left[ \frac{1}{6} (-n + n^3) \right]
        + \coeffA{m}{2} \left[ \frac{1}{30} (-n + n^5) \right]
        + \coeffA{m}{3} \left[ \frac{1}{420} (-10 n + 7 n^3 + 3 n^7) \right] - n^7 = 0
    \end{equation*}
    Multiplying by $420$ right-hand side and left-hand side, we get
    \begin{equation*}
        420 \coeffA{3}{0} n + 70 \coeffA{2}{1} (-n + n^3) + 14 \coeffA{2}{2} (-n + n^5) + \coeffA{3}{3} (-10 n + 7 n^3 + 3 n^7) - 420n^7 = 0
    \end{equation*}
    Opening brackets and rearranging the terms gives
    \begin{equation*}
        \begin{split}
            420 \coeffA{3}{0} n
            &- 70 \coeffA{3}{1} + 70 \coeffA{3}{1} n^3 - 14 \coeffA{3}{2} n + 14 \coeffA{3}{2} n^5 \\
            &- 10 \coeffA{3}{3} n + 7 \coeffA{3}{3} n^3 + 3 \coeffA{3}{3} n^7 - 420n^7 = 0
        \end{split}
    \end{equation*}
    Combining the common terms yields
    \begin{equation*}
        \begin{split}
            &n (420 \coeffA{3}{0} - 70 \coeffA{3}{1} - 14 \coeffA{3}{2} - 10 \coeffA{3}{3}) \\
            &+ n^3 (70 \coeffA{3}{1} + 7 \coeffA{3}{3})
            + n^5 14 \coeffA{3}{2}
            + n^7 (3 \coeffA{3}{3} - 420)
            = 0
        \end{split}
    \end{equation*}
    Therefore, the system of linear equations follows
    \begin{equation*}
        \begin{cases}
            420 \coeffA{3}{0} - 70 \coeffA{3}{1} - 14 \coeffA{3}{2} - 10 \coeffA{3}{3} = 0 \\
            70 \coeffA{3}{1} + 7 \coeffA{3}{3} = 0 \\
            \coeffA{3}{2} - 30 = 0 \\
            3 \coeffA{3}{3} - 420 = 0
        \end{cases}
    \end{equation*}
    Solving it, we get
    \begin{equation*}
        \begin{cases}
            \coeffA{3}{3} = 140 \\
            \coeffA{3}{2} = 0 \\
            \coeffA{3}{1} = -\frac{7}{70} \coeffA{3}{3} = -14 \\
            \coeffA{3}{0} = \frac{(70 \coeffA{3}{1} + 10 \coeffA{3}{3})}{420} = 1
        \end{cases}
    \end{equation*}
    So that odd-power identity~\eqref{eq:odd-power-identity} holds
    \begin{equation*}
        n^7 = \sum_{k=1}^{n} 140 k^3 (n-k)^3 - 14k(n-k) + 1
    \end{equation*}
    It is also clearly seen
    why the above identity is true evaluating the terms $140 k^3 (n-k)^3 - 14k(n-k) + 1$ over $0 \leq k \leq n$ as
    the OEIS sequence \href{https://oeis.org/A300785}{\textit{A300785}}~\cite{kolosov2018seventh} shows
    \begin{table}[H]
    \setlength\extrarowheight{-6pt}
    \begin{tabular}{c|cccccccc}
        $n/k$ & 0 & 1     & 2      & 3      & 4      & 5      & 6     & 7 \\
        \hline
        0     & 1 &       &        &        &        &        &       &   \\
        1     & 1 & 1     &        &        &        &        &       &   \\
        2     & 1 & 127   & 1      &        &        &        &       &   \\
        3     & 1 & 1093  & 1093   & 1      &        &        &       &   \\
        4     & 1 & 3739  & 8905   & 3739   & 1      &        &       &   \\
        5     & 1 & 8905  & 30157  & 30157  & 8905   & 1      &       &   \\
        6     & 1 & 17431 & 71569  & 101935 & 71569  & 17431  & 1     &   \\
        7     & 1 & 30157 & 139861 & 241753 & 241753 & 139861 & 30157 & 1
    \end{tabular}
    \caption{Values of $140 k^3 (n-k)^3 - 14k(n-k) + 1$.
    See the OEIS entry \href{https://oeis.org/A300785}{\texttt{A300785}}~\cite{oeis_numerical_triangle_row_sums_give_seventh_powers}.}
    \label{tab:row-sums-gives-seventh-power}
\end{table}

\end{examp}

    \begin{examp}
    Let be $m=4$ so that we have the following relation defined by (eqref)
    \begin{equation*}
        \begin{split}
            \coeffA{m}{0} n
            &+ \coeffA{m}{1} \left[ \frac{1}{6} (-n + n^3) \right]
            + \coeffA{m}{2} \left[ \frac{1}{30} (-n + n^5) \right]
            + \coeffA{m}{3} \left[ \frac{1}{420} (-10 n + 7 n^3 + 3 n^7) \right] \\
            &+ \coeffA{m}{4} \left[ \frac{1}{630} (-21 n + 20 n^3 + n^9) \right] - n^9 = 0
        \end{split}
    \end{equation*}
    Multiplying by $630$ right-hand side and left-hand side, we get
    \begin{equation*}
        \begin{split}
            630 \coeffA{4}{0} n
            &+ 105 \coeffA{4}{1} (-n + n^3) + 21 \coeffA{4}{2} (-n + n^5) \\
            &+ \frac{3}{2} \coeffA{4}{3} (-10 n + 7 n^3 + 3 n^7) + \coeffA{4}{4} (-21 n + 20 n^3 + n^9) - 630 n^9 = 0
        \end{split}
    \end{equation*}
    Opening brackets and rearranging the terms gives
    \begin{equation*}
        \begin{split}
            630 \coeffA{4}{0} n
            &- 105 \coeffA{4}{1} n + 105 \coeffA{4}{1} n ^3 - 21 \coeffA{4}{2} n + 21 \coeffA{4}{2} n^5 \\
            &- \frac{3}{2} \coeffA{4}{3} \cdot 10n + \frac{3}{2} \coeffA{4}{3} \cdot 7n^3 + \frac{3}{2} \coeffA{4}{3} \cdot 3n^7 \\
            &-21 \coeffA{4}{4} n + 20 \coeffA{4}{4} n^3 + \coeffA{4}{4} n ^9 - 630 n^9 = 0
        \end{split}
    \end{equation*}
    Combining the common terms yields
    \begin{equation*}
        \begin{split}
            &n (630 \coeffA{4}{0} - 105 \coeffA{4}{1} - 21 \coeffA{4}{2} - 15 \coeffA{4}{3} - 21 \coeffA{4}{4})  \\
            &+ n^3 \left( 105 \coeffA{4}{1} + \frac{21}{2} \coeffA{4}{3} + 20 \coeffA{4}{4} \right) + n^5 (21 \coeffA{4}{2}) \\
            &+ n^7 \left( \frac{9}{2} \coeffA{4}{3} \right) + n^9 (\coeffA{4}{4} - 630) = 0
        \end{split}
    \end{equation*}
    Therefore, the system of linear equations follows
    \begin{equation*}
        \begin{cases}
            630 \coeffA{4}{0} - 105 \coeffA{4}{1} - 21 \coeffA{4}{2} - 15 \coeffA{4}{3} - 21 \coeffA{4}{4} = 0 \\
            105 \coeffA{4}{1} + \frac{21}{2} \coeffA{4}{3} + 20 \coeffA{4}{4} = 0 \\
            \coeffA{4}{2} = 0 \\
            \coeffA{4}{3} = 0 \\
            \coeffA{4}{4} - 630 = 0
        \end{cases}
    \end{equation*}
    Solving it, we get
    \begin{equation*}
        \begin{cases}
            \coeffA{4}{4} = 630 \\
            \coeffA{4}{3} = 0 \\
            \coeffA{4}{2} = 0 \\
            \coeffA{4}{1} = - \frac{20}{105} \coeffA{4}{4} = -120 \\
            \coeffA{4}{0} = \frac{105 \coeffA{4}{1} + 21 \coeffA{4}{4}}{630} = 1
        \end{cases}
    \end{equation*}
    So that odd-power identity~\eqref{eq:odd-power-identity} holds
    \begin{equation*}
        n^9 = \sum_{k=1}^{n} 630 k^4(n-k)^4 - 120k(n-k) + 1
    \end{equation*}
\end{examp}


    \section{Finding a recurrence relation}\label{sec:finding-a-recurrence-relation}
    Another approach to determine the coefficients $\coeffA{m}{r}$ was proposed by Dr. Max Alekseyev
in MathOverflow discussion~\cite{alekseyev2018mathoverflow}.
Generally, the idea was to determine the coefficients $\coeffA{m}{r}$ recursively starting from the base case
$\coeffA{m}{m}$ up to $\coeffA{m}{r-1}, \ldots, \coeffA{m}{0}$ via previously determined values.

% begin copy paste
Before we consider the derivation of recurrent formula for coefficients $\coeffA{m}{r}$,
a few aspects regarding the Faulhaber's formula~\cite{beardon1996sums} should be discussed
\begin{equation*}
    \sum_{k=1}^{n} k^{p} = \frac{1}{p+1}\sum_{j=0}^{p} \binom{p+1}{j} \bernoulli{j} n^{p+1-j}
\end{equation*}
it is important to notice that iteration step $j$ is bounded by the value of power $p$,
while the upper index of the binomial coefficient $\binom{p+1}{j}$ is $p+1$.
Therefore, we can omit summation bounds letting $j$ run over infinity by applying
the following on the Faulhaber's formula.
\begin{align*}
    \sum_{k=1}^{n} k^{p}
    = \frac{1}{p+1}\sum_{j=0}^{p} \binom{p+1}{j} \bernoulli{j} n^{p+1-j}
    &= \left[ \frac{1}{p+1}\sum_{j=0}^{p+1} \binom{p+1}{j} \bernoulli{j} n^{p+1-j} \right] - \bernoulli{p+1} \\
    &= \left[ \frac{1}{p+1}\sum_{j} \binom{p+1}{j} \bernoulli{j} n^{p+1-j} \right] - \bernoulli{p+1}
\end{align*}
Now we are good to go through the entire derivation of the recurrent formula for
coefficients $\coeffA{m}{r}$.

By applying Binomial theorem $(n-k)^r=\sum_{t=0}^{r} (-1)^t \binom{r}{t} n^{r-t} k^t$ and Faulhaber's formula
$\sum_{k=1}^{n} k^{p} = \left[ \frac{1}{p+1}\sum_{j} \binom{p+1}{j} \bernoulli{j} n^{p+1-j} \right] - \bernoulli{p+1}$, we get
\begin{align*}
    &\sum_{k=1}^{n} k^{r} (n-k)^{r}
    = \sum_{t=0}^{r} (-1)^t \binom{r}{t} n^{r-t} \sum_{k=1}^{n} k^{t+r} \\
    &= \sum_{t=0}^{r} (-1)^t \binom{r}{t} n^{r-t} \left[ \frac{1}{t+r+1} \sum_{j} \binom{t+r+1}{j} \bernoulli{j} n^{t+r+1-j} - \bernoulli{t+r+1} \right] \\
    &= \sum_{t=0}^{r} \binom{r}{t} \left[ \frac{(-1)^t}{t+r+1} \sum_{j} \binom{t+r+1}{j} \bernoulli{j} n^{2r+1-j} - \bernoulli{t+r+1} n^{r-t} \right] \\
    &= \left[ \sum_{t=0}^{r} \binom{r}{t} \frac{(-1)^t}{t+r+1} \sum_{j} \binom{t+r+1}{j} \bernoulli{j} n^{2r+1-j}  \right]
    - \left[ \sum_{t=0}^{r} \binom{r}{t} \frac{(-1)^t}{t+r+1} \bernoulli{t+r+1} n^{r-t} \right] \\
    &= \left[ \sum_{j, t} \binom{r}{t} \frac{(-1)^t}{t+r+1} \binom{t+r+1}{j} \bernoulli{j} n^{2r+1-j}  \right]
    - \left[ \sum_{t} \binom{r}{t} \frac{(-1)^t}{t+r+1} \bernoulli{t+r+1} n^{r-t} \right]
\end{align*}
Rearranging terms yields
\begin{equation}
    \left[ \sum_{j} \bernoulli{j} n^{2r+1-j} \sum_{t} \binom{r}{t} \frac{(-1)^t}{t+r+1} \binom{t+r+1}{j}  \right]
    - \left[ \sum_{t} \binom{r}{t} \frac{(-1)^t}{t+r+1} \bernoulli{t+r+1} n^{r-t} \right]
    \label{eq:rearranging-terms}
\end{equation}
We can notice that
\begin{equation}
    \label{eq:combinatorial-identity}
    \sum_{t} \binom{r}{t} \frac{(-1)^t}{r+t+1} \binom{r+t+1}{j}
    =\begin{cases}
         \frac{1}{(2r+1) \binom{2r}r} & \text{if } j=0\\
         \frac{(-1)^r}{j} \binom{r}{2r-j+1} & \text{if } j>0
    \end{cases}
\end{equation}
An elegant proof of the binomial identity~\eqref{eq:combinatorial-identity} is presented in~\cite{scheuer2023mathstackexchange}.

In particular, equation~\eqref{eq:combinatorial-identity} is zero for $0< t \leq j$.
In order to apply~\eqref{eq:combinatorial-identity}, we have to move $j=0$ out of summation
in~\eqref{eq:rearranging-terms} to avoid division by zero in $\frac{(-1)^r}{j}$, which yields
\begin{equation*}
    \begin{split}
        \sum_{k=1}^{n} k^{r} (n-k)^{r}
        &= \frac{1}{(2r+1) \binom{2r}r} n^{2r+1} + \left[ \sum_{j \geq 1} \bernoulli{j} n^{2r+1-j} \sum_{t} \binom{r}{t} \frac{(-1)^t}{t+r+1} \binom{t+r+1}{j} \right] \\
        &- \left[ \sum_{t=0}^{r} \binom{r}{t} \frac{(-1)^t}{t+r+1} \bernoulli{t+r+1} n^{r-t} \right]
    \end{split}
\end{equation*}
Now we do not care about division by zero in $\frac{(-1)^r}{j}$ so that simplifying
above equation by using~\eqref{eq:combinatorial-identity} yields
\begin{equation*}
    \begin{split}
        \sum_{k=1}^{n} k^{r} (n-k)^{r}
        &= \frac{1}{(2r+1) \binom{2r}r} n^{2r+1}
        + \underbrace{\left[ \sum_{j \geq 1} \frac{(-1)^r}{j} \binom{r}{2r-j+1} \bernoulli{j} n^{2r-j+1} \right]}_{(\star)} \\
        &- \underbrace{\left[ \sum_{t=0}^{r} \binom{r}{t} \frac{(-1)^t}{t+r+1} \bernoulli{t+r+1} n^{r-t} \right]}_{(\diamond)}
    \end{split}
\end{equation*}
Hence, introducing $\ell=2r-j+1$ to $(\star)$ and $\ell=r-t$ to $(\diamond)$
we collapse the common terms across two sums
\begin{equation*}
    \begin{split}
        \sum_{k=1}^{n} k^{r} (n-k)^{r}
        &= \frac{1}{(2r+1) \binom{2r}r} n^{2r+1}
        + \left[ \sum_{\ell} \frac{(-1)^r}{2r+1-\ell} \binom{r}{\ell} \bernoulli{2r+1-\ell} n^{\ell} \right] \\
        &- \left[ \sum_{\ell} \binom{r}{\ell} \frac{(-1)^{r-\ell}}{2r+1-\ell} \bernoulli{2r+1-\ell} n^{\ell} \right]\\
        &= \frac{1}{(2r+1) \binom{2r}r} n^{2r+1} + 2 \sum_{\mathrm{odd \; \ell}} \frac{(-1)^r}{2r+1-\ell} \binom{r}{\ell} \bernoulli{2r+1-\ell} n^{\ell}
    \end{split}
\end{equation*}
Assuming that $\coeffA{m}{r}$ is defined by~\eqref{eq:odd-power-identity},
we obtain the following relation for polynomials in $n$
\begin{equation*}
    \sum_{r} \coeffA{m}{r} \frac{1}{(2r+1) \binom{2r}r} n^{2r+1}
    + 2 \sum_{r, \; \mathrm{odd \; \ell}} \coeffA{m}{r} \frac{(-1)^r}{2r+1-\ell} \binom{r}{\ell} \bernoulli{2r+1-\ell} n^{\ell}
    \equiv n^{2m+1}
\end{equation*}
Replacing odd $\ell$ by $k$ we get
\begin{align*}
    \sum_{r} \coeffA{m}{r} \frac{1}{(2r+1) \binom{2r}r} n^{2r+1}
    + 2 \sum_{r, \; k} \coeffA{m}{r} \frac{(-1)^r}{2r-2k} \binom{r}{2k+1} \bernoulli{2r-2k} n^{2k+1} \equiv n^{2m+1}
\end{align*}
Taking the coefficient of $n^{2m+1}$ we get
\begin{align*}
    \coeffA{m}{m} = (2m+1)\binom{2m}{m}
\end{align*}
because $\coeffA{m}{m} \frac{1}{(2m+1) \binom{2m}{m}} = 1$.

Taking the coefficient of $n^{2d+1}$ for an integer $d$ in the range $\frac{m}{2} \leq d < m$, we get
\begin{align*}
    \coeffA{m}{d} = 0
\end{align*}
because we focus on sum $2 \sum_{r, \; k} \coeffA{m}{r} \frac{(-1)^r}{2r-2k} \binom{r}{2k+1} \bernoulli{2r-2k} n^{2k+1}$,
in particular on $n^{2k+1}$ and binomial coefficient $\binom{r}{2k+1}$.
For instance, if we have to get coefficient of $n^{2d+1}$ in range $\frac{m}{2} \leq d < m$, we set $d=m-1$, thus
we have to get coefficient of $m-1$ in
$2 \sum_{r, \; k} \coeffA{m}{r} \frac{(-1)^r}{2r-2k} \binom{r}{2k+1} \bernoulli{2r-2k} n^{2k+1}$.
Therefore, we set $k=m-1$ and $r=m-1$ which leads that $\binom{r}{2k+1}=\binom{m-1}{2m-1} = 0$, so that
$\coeffA{m}{m-1} \frac{1}{(2m-1) \binom{2m-2}{m-1}} n^{2m-1} = 0$.
Same applies for every $d$ in the range $\frac{m}{2} \leq d < m$, because $r=\frac{m}{2}$ and $k=\frac{m}{2}$
means that $\binom{r}{2k+1} = \binom{\frac{m}{2}}{m+1} = 0$.

To summarize, the value of $k$ should be in range $k \leq \frac{d-1}{2}$ so that binomial coefficient $\binom{d}{2k+1}$
is non-zero.

Taking the coefficient of $n^{2d+1}$ for $d$ in the range $\frac{m}{4} \leq d < \frac{m}{2}$ we get
\begin{align*}
    \coeffA{m}{d} \frac{1}{(2d+1) \binom{2d}{d}}
    +2 (2m+1) \binom{2m}{m}\binom{m}{2d+1} \frac{(-1)^m}{2m-2d} \bernoulli{2m-2d} = 0
\end{align*}
i.e
\begin{equation*}
    \coeffA{m}{d} = (-1)^{m-1} \frac{(2m+1)!}{d!d!m!(m-2d-1)!} \frac{1}{m-d} \bernoulli{2m-2d}
\end{equation*}
Continue similarly we can compute $\coeffA{m}{r}$ for each integer $r$ in range $\frac{m}{2^{s+1}} \leq r < \frac{m}{2^{s}}$,
iterating consecutively over $s=1,2,\ldots$ by using previously determined values of $\coeffA{m}{d}$ as follows
\begin{equation*}
    \coeffA{m}{r} =
    (2r+1) \binom{2r}{r} \sum_{d \geq 2r+1}^{m} \coeffA{m}{d} \binom{d}{2r+1} \frac{(-1)^{d-1}}{d-r}
    \bernoulli{2d-2r}
\end{equation*}
Finally, we are capable to define the following recurrence relation for coefficient $\coeffA{m}{r}$
\begin{definition} (Definition of coefficient $\coeffA{m}{r}$.)
    \begin{equation}
        \label{eq:definition_coefficient_a}
        \coeffA{m}{r} =
        \begin{cases}
        (2r+1)
            \binom{2r}{r} & \mathrm{if} \; r=m \\
            (2r+1) \binom{2r}{r} \sum_{d \geq 2r+1}^{m} \coeffA{m}{d} \binom{d}{2r+1} \frac{(-1)^{d-1}}{d-r}
            \bernoulli{2d-2r} & \mathrm{if} \; 0 \leq r<m \\
            0 & \mathrm{if} \; r<0 \; \mathrm{or} \; r>m
        \end{cases}
    \end{equation}
\end{definition}
where $\bernoulli{t}$ are Bernoulli numbers~\cite{bateman1953higher}.
It is assumed that $\bernoulli{1}=\frac{1}{2}$.

For example,
\begin{table}[H]
    \begin{center}
        \setlength\extrarowheight{-6pt}
        \begin{tabular}{c|cccccccc}
            $m/r$ & 0 & 1       & 2      & 3      & 4   & 5    & 6     & 7 \\ [3px]
            \hline
            0     & 1 &         &        &        &     &      &       &       \\
            1     & 1 & 6       &        &        &     &      &       &       \\
            2     & 1 & 0       & 30     &        &     &      &       &       \\
            3     & 1 & -14     & 0      & 140    &     &      &       &       \\
            4     & 1 & -120    & 0      & 0      & 630 &      &       &       \\
            5     & 1 & -1386   & 660    & 0      & 0   & 2772 &       &       \\
            6     & 1 & -21840  & 18018  & 0      & 0   & 0    & 12012 &       \\
            7     & 1 & -450054 & 491400 & -60060 & 0   & 0    & 0     & 51480
        \end{tabular}
    \end{center}
    \caption{Coefficients $\coeffA{m}{r}$. See OEIS sequences~\cite{oeis_numerators_of_the_coefficient_a_m_r,oeis_denominators_of_the_coefficient_a_m_r}.}
    \label{tab:table_of_coefficients_a}
\end{table}


Properties of the coefficients $\coeffA{m}{r}$
\begin{itemize}
    \item $\coeffA{m}{m} = \binom{2m}{m}$
    \item $\coeffA{m}{r} = 0$ for $m < 0$ and $r > m$
    \item $\coeffA{m}{r} = 0$ for $r < 0$
    \item $\coeffA{m}{r} = 0$ for $\frac{m}{2} \leq r < m$
    \item $\coeffA{m}{0} = 1$ for $m \geq 0$
    \item $\coeffA{m}{r}$ are integers for $m \leq 11$
    \item Row sums: $\sum_{r=0}^{m} \coeffA{m}{r} = 2^{2m+1} - 1$
\end{itemize}

Let be a theorem
\begin{theorem}
    For every $n\geq 1, \; n,m\in\mathbb{N}$ there are $\coeffA{m}{0}, \coeffA{m}{1},\dots,\coeffA{m}{m}$,
    such that
    \begin{equation*}
        n^{2m+1} = \sum_{k=1}^{n}\sum_{r=0}^{m} \coeffA{m}{r} k^r (n-k)^r
        \label{eq:odd-power-theorem}
    \end{equation*}
    where $\coeffA{m}{r}$ is a real coefficient defined recursively by~\eqref{eq:definition_coefficient_a}.
\end{theorem}



    \section{Recurrence relation: examples}\label{sec:approach-via-recursion-examples}
    Consider the definition~\eqref{eq:definition_coefficient_a} of the coefficients $\coeffA{m}{r}$, it can be written as
\begin{equation*}
    \coeffA{m}{r} \colonequals
    \begin{cases}
    (2r+1)
        \binom{2r}{r}, & \text{if } r=m; \\
        \sum_{d \geq 2r+1}^{m} \coeffA{m}{d} \underbrace{(2r+1) \binom{2r}{r} \binom{d}{2r+1} \frac{(-1)^{d-1}}{d-r} \bernoulli{2d-2r}}_{T(d,r)}, & \text{if } 0 \leq r<m; \\
        0, & \text{if } r<0 \text{ or } r>m,
    \end{cases}
\end{equation*}
Therefore, let be a definition of the real coefficient $T(d,r)$
\begin{defn}
    Real coefficient $T(d,r)$
    \begin{equation*}
        T(d,r) = (2r+1) \binom{2r}{r} \binom{d}{2r+1} \frac{(-1)^{d-1}}{d-r} \bernoulli{2d-2r}
    \end{equation*}
\end{defn}
    \begin{examp}
    Let be $m=2$ so first we get $\coeffA{2}{2}$
    \begin{equation*}
        \coeffA{2}{2} = 5\binom{4}{2}=30
    \end{equation*}
    Then $\coeffA{2}{1} = 0$ because $\coeffA{m}{d}$ is zero in the range $m/2 \leq d < m$ means that zero for $d$ in
    $1 \leq d < 2$.
    Finally, the $\coeffA{2}{0}$ is
    \begin{equation*}
        \begin{split}
            \coeffA{2}{0}
            = \sum_{d \geq 1}^{2} \coeffA{2}{d} \cdot T(d, 0)
            &= \coeffA{2}{1} \cdot T(1, 0) + \coeffA{2}{2} \cdot T(2, 0) \\
            &= 30 \cdot \frac{1}{30} = 1
        \end{split}
    \end{equation*}
\end{examp}
    \begin{examp}
    Let be $m=3$ so that first we get $\coeffA{3}{3}$
    \begin{equation*}
        \coeffA{3}{3} = 7 \binom{6}{3}= 140
    \end{equation*}
    Then $\coeffA{3}{2} = 0$ because $\coeffA{m}{d}$ is zero in the range $m/2 \leq d < m$ means that zero for $d$
    in $2 \leq d < 3$.
    The $\coeffA{3}{1}$ coefficient is non-zero and calculated as
    \begin{equation*}
        \begin{split}
            \coeffA{3}{1} = \sum_{d \geq 3}^{3} \coeffA{3}{d} \cdot T(d,1) = \coeffA{3}{3} \cdot T(3,1)
            = 140 \cdot \left( -\frac{1}{10} \right) = -14
        \end{split}
    \end{equation*}
    Finally, the coefficient $\coeffA{3}{0}$ is
    \begin{equation*}
        \begin{split}
            \coeffA{3}{0}= \sum_{d \geq 1}^{3} \coeffA{3}{d} \cdot T(d,0)
            &= \coeffA{3}{1} \cdot T(1,0) + \coeffA{3}{2} \cdot T(2,0) + \coeffA{3}{3} \cdot T(3,0) \\
            &= -14 \cdot \frac{1}{6} + 140 \cdot \frac{1}{42} = 1
        \end{split}
    \end{equation*}
\end{examp}
    \begin{examp}
    Example for $\coeffA{m}{r}$ for $m=4$.
    First we get $\coeffA{4}{4}$
    \begin{equation*}
        \coeffA{4}{4} = 9 \binom{8}{4}= 630
    \end{equation*}
    Then $\coeffA{4}{2} = 0, \; \coeffA{4}{3} = 0$
    because $\coeffA{m}{d}$ is zero in the range $m/2 \leq d < m$ means that zero for $d$ in
    $2 \leq d < 4$.
    The $\coeffA{4}{1}$ coefficient is non-zero and calculated as
    \begin{equation*}
        \begin{split}
            \coeffA{4}{1}
            &= 3 \binom{2}{1} \sum_{d \geq 3} \coeffA{4}{d} \binom{d}{3} \frac{(-1)^{d-1}}{d-1} \bernoulli{2d-2} \\
            &= 3 \binom{2}{1} \coeffA{4}{4} \binom{4}{3} \frac{(-1)^3}{3} \bernoulli{6}
            = 3 \cdot 2 \cdot 630 \cdot 4 \cdot (-\frac{1}{3}) \cdot \frac{1}{42} = -120
        \end{split}
    \end{equation*}
    Finally $\coeffA{4}{0}$ coefficient is
    \begin{equation*}
        \begin{split}
            \coeffA{4}{0}
            &= 1 \binom{0}{0} \sum_{d \geq 1} \coeffA{4}{d} \binom{d}{1} \frac{(-1)^{d-1}}{d} \bernoulli{2d}
            = \sum_{d \geq 1} \coeffA{4}{d} \binom{d}{1} \frac{(-1)^{d-1}}{d} \bernoulli{2d} \\
            & = \coeffA{4}{1} \binom{1}{1} \frac{(-1)^{1-1}}{1} \bernoulli{2}
            + \coeffA{4}{2} \binom{2}{1} \frac{(-1)^{2-1}}{2} \bernoulli{4}
            + \coeffA{4}{3} \binom{3}{1} \frac{(-1)^{3-1}}{3} \bernoulli{6}
            + \coeffA{4}{4} \binom{4}{1} \frac{(-1)^{4-1}}{4} \bernoulli{8} \\
%            & = \coeffA{3}{1} \bernoulli{2} - 2 \coeffA{3}{2} \frac{1}{2} \bernoulli{4}
%            + 3 \coeffA{3}{3} \frac{1}{3} \bernoulli{6} \\
            &= \coeffA{4}{1} \frac{1}{6}
            + \coeffA{4}{2} \frac{1}{30}
            + \coeffA{4}{3} \frac{1}{42}
            + \coeffA{4}{4} \frac{1}{30} \\
            & = \frac{-120}{6} + \frac{630}{30} = 1
        \end{split}
    \end{equation*}
\end{examp}
    \begin{examp}
    Let be $m=5$ so that first we get $\coeffA{5}{5}$
    \begin{equation*}
        \coeffA{5}{5} = 11 \binom{10}{5}= 2772
    \end{equation*}
    Then $\coeffA{5}{4} = 0$ and $\coeffA{5}{3} = 0$
    because $\coeffA{m}{d}$ is zero in the range $m/2 \leq d < m$ means that zero for $d$ in $3 \leq d < 5$.
    The value of the coefficient $\coeffA{5}{2}$ is non-zero and calculated as
    \begin{equation*}
        \begin{split}
            \coeffA{5}{2}
            = \sum_{d \geq 5}^{5} \coeffA{5}{d} \cdot T(d,2) = \coeffA{5}{5} \cdot T(5,2) = 2772 \cdot \frac{5}{21} = 660
        \end{split}
    \end{equation*}
    The value of the coefficient $\coeffA{5}{1}$ is non-zero and calculated as
    \begin{equation*}
        \begin{split}
            \coeffA{5}{1}
            &= \sum_{d \geq 3}^{5} \coeffA{5}{d} \cdot T(d,1)
            = \coeffA{5}{3} \cdot T(3,1) + \coeffA{5}{4} \cdot T(4,1) + \coeffA{5}{5} \cdot T(5,1) \\
            &= 2772 \cdot \left( - \frac{1}{2} \right) = -1386
        \end{split}
    \end{equation*}
    Finally, the coefficient $\coeffA{5}{0}$ is
    \begin{equation*}
        \begin{split}
            \coeffA{5}{0}
            &= \sum_{d \geq 1}^{5} \coeffA{5}{d} \cdot T(d, 0)
            = \coeffA{5}{1} \cdot T(1, 0) + \coeffA{5}{2} \cdot T(2, 0) + \coeffA{5}{5} \cdot T(5, 0) \\
            &= -1386 \cdot \frac{1}{6} + 660 \cdot \frac{1}{30} + 2772 \cdot \frac{5}{66} = 1
        \end{split}
    \end{equation*}
\end{examp}


    \section{Conclusions}\label{sec:conclusions}
    In this manuscript, we have shown that for every $n\geq 1, \; n,m\in\mathbb{N}$
there are coefficients $\mathbf{A}_{m,0}, \mathbf{A}_{m,1}, \ldots, \mathbf{A}_{m,m}$ such that
the polynomial identity holds
\[
    n^{2m+1} = \sum_{k=1}^{n} \mathbf{A}_{m,0} k^0 (n-k)^0 + \mathbf{A}_{m,1}(n-k)^1
    + \cdots + \mathbf{A}_{m,m} k^m (n-k)^m
\]
In particular, the coefficients $\coeffA{m}{r}$ can be evaluated in both ways,
by constructing and solving a certain system of linear equations or by deriving a recurrence relation;
all these approaches are examined providing examples
in the sections~\eqref{sec:approach-via-system-of-linear-equations} and~\eqref{sec:finding-a-recurrence-relation}.
Moreover, to validate the results, supplementary Mathematica programs are available at~\cite{kolosov2023github}.


    \bibliographystyle{unsrt}
    \bibliography{PolynomialIdentityInvolvingBTandFaulhaber}
    \noindent \textbf{Version:} \texttt{Local-0.1.0}

\end{document}
